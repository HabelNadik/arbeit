\documentclass{report}
\author{Ben Haladik}
\title{Bioinformatische Anwendung von \textit{Graphlets} zur Analyse von Proteinstrukturtopologien \\ Rohfassung}
\usepackage{mathtools}
\usepackage{amsfonts}
\usepackage{ngerman}
\usepackage{hyperref}
%\usepackage{natbib}
\usepackage{rotating}
\usepackage[table, dvipsnames, usenames]{xcolor}
\usepackage{color}
\usepackage[justification=centering]{caption}
%\usepackage{algorithm2e}
%\usepackage{qtree}
%\usepackage{cite}
\definecolor{fGreen}{rgb}{0.13,0.54,0.13}

\begin{document}

\definecolor{green}{rgb}{0.13,0.54,0.13}


\maketitle

\newpage

\tableofcontents

\newpage

\chapter{Einleitung}

\section{Motivation}


Die verschiedenen Funktionen, die Proteine erf\"ullen sind zu einem gro"sen Teil durch ihre Struktur bestimmt. Diese Struktur ist ein bestimmender Faktor, f\"ur ihre F\"ahigkeit Liganden zu binden und chemische Reaktionen zu katalysieren.
Daraus folgt, dass die Analyse von Proteinstrukturen zentral f\"ur ein tieferes Verst\"andnis von zellul\"arem Leben ist. Die vergleichende Analyse von Proteinen liefert nicht nur wichtige Erkenntnise \"uber ihre Funktion, sondern hilft auch dabei, evolution\"are Verwandschaften zu entdecken, die durch reine Sequenzanalysen nicht mehr nachvollziehbar sind.

In dieser Arbeit werden Proteinstrukturtopologien untersucht. Die Topologie eines Proteins ist als die Anordnung seiner Sekund\"arstrukturelemente (SSEs) zueinander definiert. Die Betrachtung von SSEs hat den Vorteil, dass diese auch \"uber gro"se evolution\"are Distanzen stark konserviert sind. 

Zur Darstellung dieser Topologien werden hier Graphen verwendet. Graphen geh\"oren zu den am st\"arksten untersuchten mathematischen Strukturen, da sich mit ihnen viele verschiedene komplexe Zusammenh\"ange darstellen lassen. Ihre Nutzung ist \"uberall dort angebracht, wo Daten nicht als Zahlen oder Vektoren darstellbar sind, weil sie eine Menge von Objekten und ihren Beziehungen untereinander repr\"asentieren.


Graphen finden sich in der Erforschung von sozialen Netzwerken wieder und werden in der Chemie zur Darstellung von Molek\"ulen genutzt. Auch in der Bioinformatik sind Graphen das zentrale Mittel zur Darstellung komplexer Beziehungen. Interaktionen von Proteinen werden genauso als Graphen modelliert, wie Signalewege in Zellen oder eben Proteine. Die Analyse von biologischen Netzwerken ist ein zentrales Mittel, um biologische Vorg\"ange besser zu verstehen \cite{junker2011analysis}.

Der Vergleich von Graphen ist jedoch keine triviale Aufgabe. Die Frage: ''Ist dieser Graph in diesem anderen Graphen enthalten?'' befriedigend schnell beantworten zu k\"onnen, ist nicht m\"oglich, denn das zugrunde liegende Entscheidungsproblem ist NP-vollst\"andig \cite{karp1972reducibility}

Um dieses Problem zu umgehen werden hier \textit{Graphlets} angewendet. \textit{Graphlets} sind kleine, induzierte Teilgraphen, die f\"ur gro"se Graphen immer noch schnell abz\"ahlbar sind. Mit dieser Technik lassen sich in polynomieller Zeit Teilstrukturen eines Graphen ermitteln, deren Vergleich einfacher durchzuf\"uhren ist, als ein direkter Vergleich der Graphen selbst. So lassen sich \textit{Graphlets} zur vergleichenden Analyse von Proteinstrukturtopologien verwenden.




\section{\textit{State of the art}}

Mit der wachsenden Anzahl von Struktureeintr\"agen in der \textit{Protein Data Bank} (PDB) ist eine Vielzahl von Methoden entstanden, um diese Strukturen zu vergleichen.

Der wohl bekannteste Algorithmus zum Vergleich von dreidimensionalen Proteinstrukturen ist DALI von \textit{Holm} und \textit{Sander} \cite{holm1993protein}. Er f\"uhrt ein globales Alignment durch, indem Distanzmatrizen verglichen werden. In diesen Matrizen sind die intramolekularen Distanzen der C$\alpha$-Atome der jeweiligen Proteine eingetragen.

In den 23 Jahren, die seit der Ver\"offentlichung von DALI vergangen sind, sind aber noch viele weitere Methoden mit unterschiedlichen Ans\"atzen entwickelt worden. Der Algorithmus von \textit{Shindyalov} und \textit{Bourne} \cite{shindyalov1998protein} berechnet ein Alignment von Proteinstrukturen, indem er zun\"achst kleine Paare von Substrukturen aligniert und dann versucht dieses Alignment auf einen optimalen Pfad auszudehnen.

Der FATCAT-Algorithmus von \textit{Ye} und \textit{Godzik} \cite{fatcat} verwendet eine \"ahnliche Idee und versucht zus\"atzlich die Substrukturen flexibel zu alignieren, um auch gleiche Proteine mit ver\"anderter Konformation erkennen zu k\"onnen.

\textit{TM-align} von \textit{Zhang} und \textit{Skolnick} \cite{zhangtm} berechnet eine Rotationsmatrix und nutzt Dynamische Programmierung. Der Algorithmus benutzt als Bewertungsschema den sogenannten \textit{TM-Score}, der besonders geeignet ist, um lokale \"Ahnlichkeiten zu erkennen.

Der SSM-Algorithmus von \textit{Krissinel} und \textit{Hendrick} \cite{pdbefold} verwendet eine graphenbasierte Darstellung von Proteinen f\"ur ein erstes Alignment und verfeinert dieses dann durch die Berechnung der Distanzen \"aquivalenter C$\alpha$-Atome. Er wird im \textit{PDBeFold-Web-Server} implementiert. Als einziger hier beschriebener Algorithmus erm\"oglicht er schnelle multiple Strukturvergleiche \"uber einen \textit{Web-Service}.

\textit{Graphlets} wurden zuerst von \textit{Pr{\v z}ulj et al.} auf biologische Daten angewandt \cite{frqdistribution}, \cite{graphletfrequency}. Sie nutzten den \textit{Graphlet}-Algorithmus, um \"Ahnlichkeiten von Protein-Protein-Interaktionsnetzwerken zu berechnen.

\textit{N. Shervashidze} war die erste, die \textit{Graphlets} zur Analyse von Proteinen anwandte \cite{sherv_graphlets}. Sie nutzte \textit{Support Vector Machines} auf \textit{Graphlet}-Vektoren, um  f\"ur Proteingraphen zu entscheiden, ob diese Enzyme darstellen oder nicht.

\textit{Graphlets} wurden auch von \textit{Tatiana Bakirova} verwendet, um Proteinstrukturen zu analysieren \cite{bakirova2013comparison}. Sie hat das Programm \texttt{graphletAnalyser} verfasst, das in dieser Arbeit weiterentwickelt wird.

\section{Ziele}

Ziel dieser Arbeit war zun\"achst die Erweiterung der Funktionalit\"at von \\ \texttt{graphletAnalyser}. Hierzu geh\"ort zun\"achst eine funktionierende Datenbankanbindung, um die berechneten Daten abzuspeichern.
Die Suche nach markierten \textit{Graphlets} sollte so implementiert werden, dass sie auf Graphen mit beliebigen Markierungen angewandt werden kann. Deshalb wurde ein Algorithmus entwickelt und implementiert, der aus einem Alphabet von Knotenmarkierungen alle Worte berechnet, die markierte  2- und 3-\textit{Graphlets} repr\"asentieren - der \textit{Graphlet}-Worte-Algorithmus. Die Suche nach diesen markierten \textit{Graphlets} in einem Graphen wurde ebenfalls implementiert.

In Fallstudien wird \"uberpr\"uft, ob und inwiefern sich \textit{Graphlets} eignen, um die \"Ahnlichkeit von Proteinstrukturtopologien zu untersuchen. Dies wurde mit unterschiedlichen Metriken getestet. Die hierbei errechneten \"Ahnlichkeitswerte wurden mit den Ergebnissen des Strukturalignment-Programms \textit{PDBeFold} \cite{pdbefold} verglichen.

Weiterhin wurde mit dem PDBTop500-Datensatz \cite{top500} untersucht, welche strukturellen Merkmale von Proteinen mit \textit{Graphlets} ermittelt werden.

\section{Aufbau der Arbeit}

Zun\"achst wird im Kapitel \emph{Materialien und Methoden} die \textit{Protein Topology Graph Library} (PTGL) \cite{ptgl1} vorgestellt, deren Idee die Grundlage f\"ur diese Arbeit liefert. 
 
Es folgt eine Kurzbeschreibung von PLCC, der \textit{Software}, die die Graphen der PTGL erstellt und diese verwaltet, sowie eine Beschreibung des \textit{Graphlet}-Algorithmus.

Weiterhin wird das Programm \texttt{graphletAnalyser} vorgestellt, welches den \textit{Graphlet}-Algorithmus implementiert.
Anschlie"send wird die erste Metrik vorgestellt, mit denen die erhaltenen \textit{Graphlet}-Vektoren verglichen werden. Es folgt eine Beschreibung der verwendeten Datens\"atze und von \textit{PDBeFold}, dessen Ergebnisse mit denen von \texttt{graphletAnalyser} verglichen werden.

Im Ergebnisteil wird zun\"achst der modifizierte Jaccard-Index pr\"asentiert, der sich mit leichten \"Anderungen am Tanimoto-Koeffizienten orientiert. Dann folgen Beschreibungen des neuen \textit{Graphlet}-Worte-Algorithmus und der in den Fallstudien erhaltenen Ergebnisse.

Im abschlie"senden Teil \emph{Diskussion und Ausblick} wird versucht, die Frage zu kl\"aren, ob sich \textit{Graphlets} f\"ur multiplen Proteinstrukturvergleich und \"ahnliche Anwendungen eignen. Weiterhin wird untersucht, ob Modifikationen der \textit{Graphlet}-Vektoren, oder der Metriken n\"otig sein k\"onnten, um bessere Ergebnisse zu erhalten. 



\chapter{Materialien und Methoden}

%TODO: Einleitung des MatMeth-Teils \"uberarbeiten


%Um die Proteinstrukturtopologien aus der PTGL zu vergleichen wurde das Programm \texttt{graphletAnalyser} genutzt und erweitert. Es wurde bereits 2013 von \textit{Tatiana Bakirova} im Rahmen ihrer Diplomarbeit im Arbeitskreis \textit{Molekulare Bioinformatik} geschrieben. Die urspr\"ungliche Funktionalit\"at wurde erweitert. Hierbei wurden Funktionen zur Analyse von Komplexgraphen, Aminos\"auregraphen und den Sekund\"arstrukturgraphen implemetiert. Diese Graphen stammen allesamt aus der PTGL (\underline{P}rotein \underline{T}opology \underline{G}raph \underline{L}ibrary) von Tim Sch\"afer.


\section{PTGL}


Die \underline{P}rotein \underline{T}opology \underline{G}raph \underline{L}ibrary entstand aus einer Idee von \textit{Patrick May} und \textit{Ina Koch} (\cite{ptgl1}). Ausgehend von der Tatsache, dass sich Proteinstrukturtopologien als r\"aumliche Beziehungen von SSEs untereinander definieren lassen, verwendet die PTGL \emph{Graphen}, um Proteinstrukturtopologien darzustellen.
Hierbei stellen die Knoten des Graphen die SSEs eines Proteins dar. Sie werden dem jeweiligen SSE entsprechend markiert. Knoten, die $\alpha$-Helices repr\"asentieren, werden mit einem H markiert, $\beta$-Faltbl\"atter mit einem E. Weiterhin erm\"oglicht die PTGL die Darstellung von Liganden, (\cite{vplg}) denen mit L markierte Knoten zugeordnet werden. Um die r\"aumliche Nachbarschaft von Sekund\"arstrukturen und Liganden mit- und untereinander darstellen zu k\"onnen, werden ungerichtete Kanten zwischen Knoten gezogen, wenn die entsprechenden Elemente benachbart sind. Jede Polypeptidkette eines Proteins wird dann als \emph{Proteingraph} dargestellt. Die Zusammenhangskomponenten eines Proteingraphen werden als Faltungsgraphen bezeichnet, weil sie typischerweise eine unabh\"angige Faltungseinheit darstellen.

Durch diese abstrahierte Darstellung k\"onnen zentrale Charakteristika eines Proteins wie Motive und Dom\"anen einfach visualisiert werden.


\section{PLCC}

PLCC ist die \textit{Software}, die die Daten der PTGL generiert und verwaltet. Sie wird von \textit{Tim Sch\"afer} geschrieben und verwaltet.


\paragraph{Die Berechnung der Graphen der PTGL}

erfolgt unter Verwendung der entsprechenden PDB und DSSP-Dateien. Um den Graphen f\"ur eine Polypeptidkette zu berechnen, werden aus der DSSP-Datei die SSEs des Proteins ausgelesen. F\"ur jedes Paar von SSEs wird die Anzahl der r\"aumlichen Kontakte ihrer Residuen in der PDB-Datei berechnet. Wenn die Anzahl dieser Kontakte einen gewissen Grenzwert \"uberschreitet, wird angenommen, dass diese SSEs r\"aumlich benachbart sind und die jeweiligen Knoten werden durch eine Kante verbunden. So wird f\"ur jede Polypeptidkette ein Graph erstellt.

\paragraph{Komplexgraphen}

werden ebenfalls in dieser Arbeit untersucht. Ihre Berechnung erfolgt analog zur Berechnung der Proteingraphen. Der Unterschied besteht darin, dass ein Komplexgraph mehrere Polypeptidketten beschreibt.

\paragraph{Aminos\"auregraphen} werden analog zu Proteingraphen und Komplexgraphen berechnet. Der Unterschied zu den anderen Graphformaten ist, dass keine SSEs betrachtet werden. Stattdessen repr\"asentiert jeder Knoten eine Aminos\"aure eines Proteins. Die Knoten werden entsprechend der chemischen Eigenschaften der Aminos\"auren markiert. Knoten, die saure oder basische Residuen darstellen, werden mit einem c markiert. Ein p markiert Knoten f\"ur polare Residuen, die weder sauer noch basisch sind. F\"ur unpolare Aminos\"auren wird ein h verwendet. Auch Liganden k\"onnen in Aminos\"auregraphen dargestellt werden. Ihre Knoten werden durch ein ? markiert. Aminos\"auregraphen k\"onnen Proteinkomplexe und einzelne Polypeptidketten darstellen.


Weiterhin erm\"oglicht PLCC den Vergleich von \textit{Graphlet}-Vektoren, die im folgenden Teil vorgestellt werden.


\section{Der \textit{Graphlet}-Algorithmus}


Um diese Graphen vergleichen zu k\"onnen, werden \textit{Graphlets} verwendet. Diese sind im Gegensatz zu Graph-Isomorphismen in polynomieller Zeit berechenbar \cite{sherv_graphlets}. \textit{Graphlets} sind kleine, induzierte Teilgraphen mit bis zu 5 Knoten. Die Abbildungen \ref{fig:3graphlets}, \ref{fig:4graphlets} und \ref{fig:5graphlets} zeigen die \textit{Graphlets} mit 3, 4 und 5 Knoten. Wir betrachten hierbei nur die zusammenh\"angenden \textit{Graphlets}. Sie werden durch den Algorithmus gez\"ahlt und ihre jeweilige Anzahl wird in einen Vektor geschrieben, den wir als \textit{Graphlet}-Vektor bezeichnen.


\subsection{Beschreibung des Algorithmus}
Um alle \textit{Graphlets} der Gr\"o"se $k \in \{3,4,5\}$ zu z\"ahlen, werden alle Euler-Wege der L\"ange $k-1$ in dem gegebenen Graphen gesucht. F\"ur jeden gefundenen Euler-Weg werden alle inzidenten Kanten aller Knoten des Weges \"uberpr\"uft. Wird hierbei ein \textit{Graphlet} gefunden, wird der Z\"ahler f\"ur das entsprechende \textit{Graphlet} im \textit{Graphlet}-Vektor um den Wert an der entsprechenden Stelle der unten stehenden Gewichtungsvektoren $w_k$ erh\"oht.


F\"ur \textit{Graphlets} mit $k \geq 4$ existieren zus\"atzlich die sogenannten Stern-\textit{Graphlets} ($g_4$ in \ref{fig:4graphlets}, sowie $g_{19}$,$g_{20}$ und $g_{21}$ in \ref{fig:5graphlets}), die keinen solchen Euler-Weg enthalten.

\begin{subequations}
\label{eq:w-vector}
\begin{align}
\text{\textit{Graphlet}-Gewichtungsvektoren} \\
w_2 := \left( \frac{1}{2} \right) \\
w_3 := \left( \frac{1}{6}, \frac{1}{2} \right) \\
w_4 := \left( \frac{1}{24}, \frac{1}{12}, \frac{1}{4}, 1, \frac{1}{8},\frac{1}{2} \right) \\
w_5 := \left( \frac{1}{120}, \frac{1}{72}, \frac{1}{48}, \frac{1}{36}, \frac{1}{28}, \frac{1}{20}, \frac{1}{14}, \frac{1}{10}, \frac{1}{12}, \frac{1}{8}, \frac{1}{4}, \frac{1}{2}, \frac{1}{12} \frac{1}{12}, \frac{1}{4} \frac{1}{4}, \frac{1}{2}, 1, \frac{1}{2}, 1\right)
\end{align}
\end{subequations}

Jede Stelle eines Gewichtungsvektors $w_k$ ist mit einem \textit{Graphlet} assoziiert. Die Nenner der Br\"uche in den Vektoren entsprechen den Anzahlen der Euler-Wege der L\"ange $k-1$ in den entsprechenden \textit{Graphlets}. Wenn alle Euler-Wege eines \textit{Graphlets} besucht wurden und der zugeh\"orige induzierte Teilgraph ihm entspricht, wird sein Z\"ahler im \textit{Graphlet}-Vektor ganzzahlig und das \textit{Graphlet} gilt als gefunden.
Hierbei bilden  die Stern-\textit{Graphlets}, die keine Euler-Wege der L\"ange $k-1$ enthalten, nat\"urlich die Ausnahme.
 
Da wir 29 verschiedene \textit{Graphlets} betrachten, hat ein \textit{Graphlet}-Vektor 29 Stellen. F\"ur den Vergleich von Proteinstrukturen bedeutet dies, dass der Berechnugsaufwand in $O(1)$ liegt, da anstatt von Graphen, oder dreidimensionalen K\"orpern Vektoren im Vektorraum $\mathbb{R}^{29}$ verglichen werden.

TODO: Pseudocode in Anhang einf\"ugen


\section{\texttt{graphletAnalyser}}

In seiner urspr\"unglichen Version wurde das Programm bereits 2013 von \textit{Tatiana Bakirova} geschrieben. Im Sommer 2015 wurde es um einige Funktionen erweitert.

\texttt{graphletAnalyser} ist in C++ geschrieben. Zur internen Darstellung der Graphen wird die \textit{Boost-Graph-Library} verwendet. Die Datenbankanbindug wird mittels der Bibliothek pqxx realisiert.

\paragraph{Als Input} erh\"alt das Programm eine oder mehrere GML-Dateien, die Graphen darstellen. Aus jeder GML-Datei wird mit der \textit{Boost-Graph-Library} intern ein Graph erstellt, der f\"ur die Berechnungen verwendet wird.

F\"ur den eingelesenen Graphen berechnet das Programm alle \textit{Graphlets} der Gr\"o"sen 3,4 und 5. Die Implementierung entspricht der \textit{Matlab}-Implementierung von \textit{N. Shervashidze} \cite{sherv_graphlets}. Zus\"atzlich berechnet es markierte \textit{Graphlets} der Gr\"o"sen 2 und 3, wobei die Markierungen den SSE-Zuordnungen der PTGL entsprechen. Es gibt 35 verschiedene markierte \textit{Graphlets} mit 2 oder 3 Knoten. Unter Ber\"ucksichtigung dieser Markierungen erh\"oht sich die Anzahl der Stellen der \textit{Graphlet}-Vektoren auf 64.
Die Knotengradverteilungen k\"onnen ebenfalls berechnet werden.

\paragraph{Der Output} des Programms umfasst zum einen die Ausgabe der \textit{Graphlet}-Vektoren mit oder ohne markierte \textit{Graphlets} in den Formaten csv, matlab und nova. Zus\"atzlich k\"onnen die \textit{Graphlet}-Vektoren in einer Datenbank, die zuvor mit PLCC erstellt wurde gespeichert werden. Die Knotengradverteilung kann zusammen mit anderen Daten eines Graphen als .csv-Datei ausgegeben werden.


\section{\"Ahnlichkeitsma"s}

Durch die Verwendung der \textit{Graphlets} wird der direkte Vergleich von Graphen durch einen Vergleich von \textit{Graphlet}-Vektoren ersetzt. Typischerweise wird eine eine Abstandsmessung durchgef\"uhrt, um Vektoren zu vergleichen. Hier verwenden wir eine Metrik von \textit{Pr\v{z}ulj et al.}.



\subsection{Relative \textit{Graphlet}-H\"aufigkeiten-Distanz}

\textit{Pr\v{z}ulj et al.} haben \textit{Graphlets} bereits auf Protein-Protein-Interaktionsnetzwerke (\cite{frqdistribution}) angewandt. Als Ma"s f\"ur die \"Ahnlichkeit von Netzwerken nutzen sie die Relative-\textit{Graphlet}-H\"aufigkeiten-Distanz (RGF) . Diese Metrik berechnet den Abstand $D(G,H)$ zwischen zwei Graphen $G$ und $H$ als logarithmierte Differenz der normalisierten Anzahl der \textit{Graphlets} in $G$ und $H$. Sie ist folgenderma"sen definiert: \\

Sei $N_{i}(G)$ die Anzahl der \textit{Graphlets} von Typ $i \in {1,...,29}$ und \\ $T(G) = \sum_{i = 1}^{29} N_{i}(G)$ die Anzahl der \textit{Graphlets} in $G$, beziehungsweise $H$.\\

Dann ist $D(G,H)$ f\"ur zwei Graphen $G$ und $H$ definert als:

\begin{subequations}
\begin{align}
D(G,H) := \sum_{i = 1}^{29} | F_{i}(G) - F_{i}(H) | \\
mit F_{i}(G) := - log(\frac{N_{i}(G)}{T(G)})
\end{align}
\end{subequations}



Diese Metrik l\"asst sich analog zur euklidischen Distanz auffassen, denn sie berechnet den Abstand f\"ur zwei Vektoren $x,y$ als Differenz ihrer Eintr\"age $x_i -y_i$. Sie verwendet die normalisierten \textit{Graphlet}-Vektoren unter der Annahme, dass die \"Ahnlichkeit zweier Netzwerke sich aus der \"Ahnlichkeit lokaler Substrukturen ableiten l\"asst \cite{frqdistribution}. Somit k\"onnen Netzwerke, die \"ahnliche Substrukturen habe, sich aber in ihrer Gr\"o"se stark unterschieden, immer noch als \"ahnlich erkannt werden. Zus\"atzlich soll durch diese Normalisierung ausgeschlossen werden, dass einzelne besonders h\"aufige \textit{Graphlets} in einem Graphen die Messung verzerren.

Da alle Eintr\"age der Vektoren aus dem Intervall $ [0,1] \in \mathbb{R} $ stammen werden sie logarithmiert. Damit erh\"alt man Abst\"ande, die aus deutlich gr\"o"seren Intervallen stammen. Die Ergebnisse k\"onnen so leichter interpretiert werden und der Einfluss der von Rundungsfehlern verkleinert sich.
 
Weiterhin wurde gezeigt, (\cite{frqdistribution}) dass diese Metrik auch bei verrauschten Daten noch sehr gut funktioniert. Hierbei ist jedoch zu beachten, dass sie bisher vor allem f\"ur sehr gro"se Netzwerke mit mehreren Tausend Knoten und Kanten verwendet wurde. Diese Gr\"o"se kann von Aminos\"auregraphen erreicht werden, wenn sie gro"se Proteinkomplexe modellieren. Proteingraphen und Komplexgraphen sind aber deutlich kleiner.

\section{\textit{PDBeFold}}

Im Rahmen der Fallstudien wird \texttt{graphletAnalyser} mit \textit{PDBeFold} \cite{pdbefold} verglichen. \textit{PDBeFold} f\"uhrt einen graphenbasierten Strukrutvergleich durch. Analog zur PTGL modellieren Knoten SSEs. Zus\"atzlich werden diese mit der Anzahl von Residuen der SSE markiert. Im Gegensatz zur PTGL verbindet PDBeFold jeden Knoten durch eine Kante mit jedem anderen Knoten. Die Kanten werden hierbei mit Vektoren markiert, in denen unter anderem die Distanzen und Winkel zwischen den SSEs eingetragen sind.

Die Proteine werden also als vollst\"andige, ungerichtete Graphen modelliert, in denen jede Kante alle Informationen zur r\"aumlichen Orientierung ihrer inzidenten Knoten enth\"alt. Diese Graphen werden vorberechnet.

Zum Vergleich von Strukturen wird f\"ur diese Graphen zun\"achst ein Alignment durchgef\"uhrt, um eine Absch\"atzung der strukturellen \"Ahnlichkeit zu erhalten und die \"aquivalenten Residuen zu ermitteln. Im Anschluss f\"uhrt PDBeFold ein Alignment der C$\alpha$-Atome durch und berechnet die \"Ahnlichkeit mit der \textit{Root-Mean-Square-Deviation}.


\section{Datens\"atze}

Im folgenden Teil sind die verwendeten Datens\"atze beschrieben. Es wurden zwei Fallstudien mit Abstandsmessungen durchgef\"uhrt, um diese zu testen.

Zus\"atzlich wurden \textit{Graphlet}-Vektoren f\"ur einen Datensatz aus 500 nicht redundanten PDB-Eintr\"agen berechnet. F\"ur diesen Datensatz wurden keine Abstandsmessungen durchgef\"uhrt, stattdessen wurden die \textit{Graphlet}-Vektoren selbst analysiert.


\subsection{Fallstudien  - Datensatz 1}


Dieser Datensatz enth\"alt 15 verschiedene Proteine mit bekannten strukturellen \"Ahnlichkeiten. Die Proteine wurden so gew\"ahlt, dass ihre Einordnungen in strukturelle Klassen in den Datenbanken CATH und SCOPe \"aquivalent zueinander sind.
Der Datensatz wurde also so zusammengestellt, dass bez\"uglich der strukturellen Einteilung der Proteine der gr\"o"stm\"ogliche Konsens besteht.
Je 5 Proteine dieses Datensatzes befinden sich in der gleichen Klasse. Von diesen 5 besitzen je 4 die gleiche Topologie und von diesen entstammen wiederum 3 der gleichen homologen Superfamilie. Von diesen ist ein Paar direkt homolog mit einer Sequenzidentit\"at von mehr als 95\%. Das dritte Protein ist zu den anderen beiden entfernt homolog mit einer Sequenzidentit\"at von weniger als 30\%.

Es sind Proteine der Klassen $\alpha$, $\beta$ und $\alpha/\beta$ vertreten. Proteine mit wenigen SSEs wurden nicht betrachtet, da f\"ur diese die Proteingraphen und die Komplexgraphen in den meisten F\"allen keine Kanten besitzen. Dadurch k\"onnen keine \textit{Graphlets} in diesen Graphen gefunden werden.

Weiterhin wurde darauf geachtet, dass jedes Protein dieses Datensatzes aus genau einer Polypeptidkette mit genau einer Dom\"ane besteht. Dies hat mehrere Gr\"unde.

Zum ersten erm\"oglicht diese Auswahl die beste Vergleichbarkeit der Messungen mit Proteingraphen, Komplexgraphen und Aminos\"auregraphen, da Proteingraphen im Gegensatz zu Komplexgraphen und Amins\"auregraphen nur eine Polypeptidkette darstellen k\"onnen. Durch die Beschr\"ankung auf Proteine mit genau einer Polypeptidkette wird ein direkter Vergleich durch die Korrelation der paarweisen Distanzen in den unterschiedlichen Formaten m\"oglich.

Ein weiterer Grund f\"ur die Beschr\"ankung auf solche Proteine ist, dass man ein ''klareres Signal'' erh\"alt. Die Struktur von Proteinen wird haupts\"achlich anhand ihrer Dom\"anen beschrieben. Wenn \textit{Graphlets} sich wirklich f\"ur den Vergleich von Topologien eignen, dann m\"ussten gleiche Dom\"anen zu gleichen, oder zumindest \"ahnlichen \textit{Graphlet}-Vektoren f\"uhren.
Da diese Vektoren globale Struktureigenschaften des Proteins beschreiben erhielte man f\"ur ein Multidom\"anenprotein mit unterschiedlichen Dom\"anen einen Vektor, der die Struktureigenschaften beider Dom\"anen ''vermischt'' und das Signal verrauscht.


In der folgenden Tabelle \ref{tab:set1} sind die PDB-IDs der Proteine dieses Datensatzes zusammen mit ihrer CATH-ID und der Klassifizierung aus der PDB Datei aufgef\"uhrt.



\begin{table}[h!]
\label{tab:set1}
\begin{tabular}{ | c || c | c |}

\hline
PDB-ID & CATH-ID        & Klassifizierung      \\ \hline
1qpu   & 1.20.120.10 & Elektronentransport     \\ \hline
1qq3   & 1.20.120.10 & Elektronentransport     \\ \hline
1cgn   & 1.20.120.10 & Elektronentransport     \\ \hline
1he9   & 1.20.120.260& Toxin (Exoenzym)        \\ \hline
3gf9   & 1.20.900.10 & Endocytose              \\ \hline
1exs   & 2.40.128.20 & Lipid bindendes Protein \\ \hline
1ngl   & 2.40.128.20 & Transport Protein       \\ \hline
1qqs   & 2.40.128.20 & Zucker bindendes Protein\\ \hline
3slo   & 2.40.128.130& Protein Transport       \\ \hline
1wjx   & 2.40.280.10 & RNA-bindendes Protein   \\ \hline
5chy   & 3.40.50.2300& Signaltransduktion      \\ \hline
2id9   & 3.40.50.2300& Signal Protein          \\ \hline
3i42   & 3.40.50.2300& unbekannte Funktion     \\ \hline
1d4o   & 3.40.50.1220& Oxidoreductase          \\ \hline
2w0i   & 3.40.20.10  & Transferase             \\ 
\hline
\end{tabular}
\caption{PDB-Eintr\"age der ersten Fallstudie. Die PDB-IDs sind mit dem CATH-Code der zugeh\"origen homologen Superfamilie und der Klassifizierung aus dem \textit{Header} der PDB-Datei in der entsprechenden Zeile eingetragen}
\end{table}

F\"ur diesen Datensatz wurden alle \textit{Graphlet}-Vektoren berechnet. Die paarweisen Distanzen dieser Vektoren zueinander wurden mit der RGF und dem modifizierten Jaccard-Index berechnet. F\"ur die berechneten Distanzmatrizen wurden die Korrelationen der Distanzen in den verschiedenen Formaten berechnet und sie wurden mit den von PDBeFold berechneten Distanzen verglichen. 


\subsection{Fallstudien - Datensatz 2}

F\"ur die zweite Fallstudie wurden Proteine ausgew\"ahlt, die weniger engen Beschr\"ankungen unterliegen. Die meisten von ihnen besitzen mehrere - h\"aufig unterschiedliche - Dom\"anen und mehrere Polypeptidketten. Sie wurden so ausgew\"ahlt, dass es zu jedem Protein mindestens ein anderes gibt, das die selbe Topologie hat, um festzustellen, ob ein solches Paar auch immer die niedrigsten Distanzen erh\"alt.

Weiterhin wurde darauf geachtet, diese Proteine so auszuw\"ahlen, dass die Analyse durch \textit{Graphlets} besonders schwierig ausf\"allt. Es sind Proteine aus der Familie der \textit{TIM-Barrels} enthalten, deren Strukturen sich stark unterscheiden, auch wenn sie als homolog angesehen werden und es wurden Paare von PDB-Strukturen eingef\"ugt, die sich nur durch eine Konformations\"anderung unterscheiden.





\begin{tabular}{ | c || c | c | }

\hline
PDB-ID & CATH-ID  \\ \hline
2UTG & 1.10.210.10  \\ \hline % uteroglobin - pdb-klasse: steroid bindend
1VIB & 1.10.287.120 \\ \hline % neurotoxin - pdb-klasse: neurotoxin
1QPU & 1.20.120.10  \\ \hline % Cytochrom b
1QQ3 & 1.20.120.10  \\ \hline % cytochrom b oxidiert
2XL6 & 1.20.120.10  \\ \hline % cytochrom c mit NO
2YL1 & 1.20.120.10  \\ \hline% cytochrom c mit CO
1rxq & 1.20.120.450 \\ \hline% metall abhaengige hydrolase - pdb-head: metall abhaengiges protein
2qe9 & 1.20.120.450 \\ \hline% metall abhaengige hydrolase - pdb-head: hydrolase
2RD9 & 1.20.120.450 \\ \hline% metall abhaengige hydrolase - pdb-head: hydrolase
1M4R & 1.20.1250.10 \\ \hline% interleukin 22 - pdb-head: cytokin
1HGU & 1.20.1250.10 \\ \hline% wachstumshormon - pdb-head: hormon
1PV6 & 1.20.1250.20 \\ \hline% lactose permease - pdb-head: transportprotein
1AWR & 2.40.100.10  \\ \hline% enzymkomplex cypa und hagpia ?
1D9C & 1.20.1250.10 \\ \hline% BOVINE INTERFERON-GAMMA
1NGL & 2.40.128.20  \\ \hline% transportprotein
1N7V & 2.105.10.10  \\ \hline% riesiges Virusprotein - rezeptor bindendes protein
7TIM & 3.20.20.70   \\ \hline% triosephosphat isomerase phosphat-komplex
1NEY & 3.20.20.70   \\ \hline% triosephosphat isomerase DHAP-komplex
2V5L & 3.20.20.70   \\ \hline%  TRYPANOSOMAL TRIOSEPHOSPHATE ISOMERASE - neue kristallform
1V3Z & 3.30.70.100  \\ \hline% Acylphosphatase
1A17 & 1.25.40.10   \\ \hline% pdb-header sagt hydrolase - muesste anderen hydrolasen aehnlich sein
1ar0 & 3.10.450.50  \\ \hline% Nuclear transport factor 2

\end{tabular}







\subsection{Der PDBTop500-Datensatz}

Der PDBTop500-Datensatz wurde von \textit{Lovell et al.} vorgestellt (\cite{top500}). Diese nutzten ihn zur Validierung von PDB-Strukturen mittels C$\alpha$-Geometrie.
Der Datensatz enth\"alt 500 nicht redundante PDB-Eintr\"age mit einer Aufl\"osung von 1,8 \AA \\
oder besser.

Er wurde ausgew\"ahlt, weil die \textit{Graphlet}-Vektoren f\"ur ihn schnell berechenbar waren. Im Gegensatz zu anderen Datens\"atzen mit nicht redundanten PDB-Eintr\"agen wie dem FATCAT- oder dem ASTRAL-Datensatz enth\"alt dieser keine riesigen Proteinkomplexe, deren \textit{Graphlets} auf durchschnittlichen Computern nicht berechnbar sind. 
Er ist aber immer noch gro"s genug, um Strukturvergleichsmethoden validieren zu k\"onnen. \textit{Zhang} und \textit{Skolnick} verwendeten zur Validierung von \textit{TM-align} einen Datensatz aus 200 PDB-Eintr\"agen (\cite{zhangtm}).

Da ein Vergleich von Strukturvergleichsmethoden auf einem solchen Datensatz zu aufw\"andig war, wurde dieser Datensatz f\"ur eine statistische Analyse der \textit{Graphlet}-Vektoren genutzt.

F\"ur die relativen H\"aufigkeiten der einzelnen \textit{Graphlets} wurden Minima, Maxima, Varianzen und Durchschnittswerte berechnet. Hierbei war das Ziel festzustellen, ob wirklich alle \textit{Graphlets} f\"ur Strukturvergleiche notwendig sind.


\chapter{Ergebnisse}

\section{Der modifizierte Jaccard-Index}


Der Jaccard-Index ist im eigentlichen Sinne ein Ma"s, um die \"Ahnlichkeit von gleichm\"achtigen Mengen zu bewerten. F\"ur zwei Mengen $A,B$ berechnet sich der Jaccard-Index $D_{Jac}(A,B)$ folgenderma"sen:

\[ D_{Jac}(A,B) := \frac{\sum_{x \ in A \land x \in B} 1}{\sum_{x \in A \lor x \in B} 1} \]

Dementsprechend sind zwei Mengen $A,B$ gleich, wenn  gilt $D_{Jac} = 1$ und disjunkt, wenn gilt $D_{Jac} = 0$. Mit ihm wird die relative Anzahl der Elemente beider Mengen berechnet.
Um dieses Ma"s in sinnvoller Weise auf \textit{Graphlet}-Vektoren zu \"ubertragen wurde ein zus\"atzlicher Faktor $k \in \mathbb{R} $ mit $k \in [0,1]$  eingef\"uhrt, der als Pr\"azisionsfaktor zu verstehen ist. Die Defintion des modifizierten Jaccard-Index $D_{Jac-m}(v,w)$ f\"ur zwei Vektoren $v,w$ lautet also:

\[ D_{Jac-m}(v,w) := \frac{\sum_{i = 1}^n x_i}{\sum_{x \in A \lor x \in B} 1} \]

Hierbei gilt:

\[ x_i = 
   \begin{cases}
     1     & \quad \mathrm{if} \quad v_i >= w_i \times k \land w_i >= v_i \times k \\
     0     & \quad \mathrm{else} \\
   \end{cases}
\]

In dieser modifizierten Variante werden zwei Vektoren $v,w$ als gleich angesehen, wenn sich $v_i,w_i$ f\"ur alle $i$ h\"ochstens um den Faktor $k$ unterschieden.
Dies steht im Gegensatz zur RGF, die - analog zu einer Vektornorm - die Abst\"ande zwischen zwei \textit{Graphlet}-Vektoren misst.


\section{Der \textit{Graphlet}-Worte-Algorithmus}

In der letzten Version von \texttt{graphletAnalyser} war es bereits m\"oglich markierte \textit{Graphlets} mit 2 und 3 Knoten in Proteingraphen zu z\"ahlen. Diese Funktionalit\"at wurde im Rahmen dieser Arbeit verallgemeinert, so dass der Nutzer beliebige Alphabete angeben kann.
Der Algorithmus erh\"alt das Alphabet $\sum = \{ \sigma_i : i \in \mathbb{N} \}$ der Knotenmarkierungen. Aus diesem Alphabet berechnet er Worte $w$, die zur Repr\"asentation der markierten \textit{Graphlets} genutzt werden.
Hierbei k\"onnen verschiedene Worte das gleiche \textit{Graphlet} repr\"asentieren. Im Falle von 2-\textit{Graphlets} repr\"asentieren die zwei Worte $(\sigma_i,\sigma_j)$ und $(\sigma_j, \sigma_i)$ das gleiche markierte \textit{Graphlet} mit den Knotenmarkierungen $\sigma_i,\sigma_j$. Worte, die das gleiche \textit{Graphlet} repr\"asentieren, werden im Folgenden als \emph{\"aquivalente Graphlet-Worte}  bezeichnet.


Die Berechnung der \"aquivalenten \textit{Graphlet}-Worte der L\"ange 2 ist trivial. Aus dem Alphabet $\sum$ werden alle Worte $w = (\sigma_i, \sigma_j)$ berechnet, wobei Spiegelungen nicht mit ausgegeben werden, da zwei Worte $ (\sigma_j, \sigma_i) $ und $ (\sigma_j, \sigma_i) $ \"aquivalente \textit{Graphlet}-Worte sind.

Die Berechnung aller \"aquivalenten \textit{Graphlet}-Worte der L\"ange 3 ist komplizierter, da sie f\"ur zwei verschiedene \textit{Graphlets} berechnet werden m\"ussen. F\"ur das \textit{Graphlet} $g_1$ sind alle Worte \"aquivalent zueinander, die zyklische Vertauschungen voneinander sind. F\"ur das \textit{Graphlet} $g_2$ sind Worte \"aquivalent zueinander, die Spiegelungen voneinander sind (siehe Abbildung \ref{fig:3graphlets}). \\



%\begin{algorithmic}
% INPUT: Ein Alphabet \sum = \{ \sigma_1, ... , \sigma_n \}
% OUTPUT: Zwei Listen Wortliste_3-Weg, Wortliste_3-Kreis \sub \sum*
% Liste Wortliste_3-Weg
% Liste Wortliste_3-Kreis
%\For{ i = 1 to n}
% Wortliste.add(\sigma_i \sigma_i \sigma_i)
%\For{ k = i + 1 to n}
% Wortliste_3-Weg.add(\sigma_i \sigma_k \sigma_i)
% Wortliste_3-Weg.add(\sigma_i \sigma_i \sigma_k)
% Wortliste_3-Weg.add(\sigma_i \sigma_k \sigma_k)
% Wortliste_3-Weg.add(\sigma_k \sigma_i \sigma_k)
%
% Wortliste_3-Kreis.add(\sigma_i \sigma_k \sigma_i)
% Wortliste_3-Kreis.add(\sigma_i \sigma_k \sigma_k)
%\For{ m = k + 1 to n}
% Wortliste_3-Weg.add(\sigma_i \sigma_k \sigma_m)
% Wortliste_3-Weg.add(\sigma_m \sigma_i \sigma_k)
% Wortliste_3-Weg.add(\sigma_k \sigma_m \sigma_i)
%
% Wortliste_3-Kreis.add(\sigma_i \sigma_k \sigma_m)
% Wortliste_3-Kreis.add(\sigma_m \sigma_k \sigma_i)
%\EndFor
%\EndFor
%\EndFor
%\end{algorithmic}

Pseudocode Platzhalter \\
Pseudocode Platzhalter \\
Pseudocode Platzhalter \\
Pseudocode Platzhalter \\
Pseudocode Platzhalter \\
Pseudocode Platzhalter \\
Pseudocode PLatzhalter \\
Pseudocode PLatzhalter \\
Pseudocode PLatzhalter \\
Pseudocode PLatzhalter \\
Pseudocode PLatzhalter \\
Pseudocode PLatzhalter \\
Pseudocode PLatzhalter \\
Pseudocode PLatzhalter \\
Pseudocode PLatzhalter \\
Pseudocode PLatzhalter \\
Pseudocode PLatzhalter \\
Pseudocode PLatzhalter \\
Pseudocode PLatzhalter \\
Pseudocode PLatzhalter \\
Pseudocode PLatzhalter \\
Pseudocode PLatzhalter \\
Pseudocode PLatzhalter \\


Der Algorithmus besteht aus 3 \textit{for}-Schleifen, die \"uber das Alphabet iterieren. In der \"au"sersten Schleife werden Worte hinzugef\"ugt, in denen alle Buchstaben gleich sind.
In der zweiten Schleife wird jeweils der n\"achste Buchstabe des Alphabets betrachtet. F\"ur jedes Paar $a, b$ von Buchstaben \"uber einem Alphabet $\sum$ sind die Mengen der \"aquivalenten Worte f\"ur das \textit{Graphlet} $g_1$ $P_{3-Kreis} = \{ \}$ und $P_{3-Weg} = \{ aaa, aba, aab, abb  \}$ f\"ur das \textit{Graphlet} $g_2$

TODO: Beschreibung der Mengen, Beweis



F\"ur das Alphabet der SSE- und Komplexgraphen $ \sum_{SSE} := \{ H, E, L \} $ und das Alphabet der Aminos\"auregraphen $ \sum_{AA} := \{ h, p, c, ? \} $ gibt der oben beschriebene Algorithmus die folgenden Listen aus:


\begin{subequations}
\begin{align}
p_2 := (HH, HE, HL, EE, EL, LL) \\
p_{3-Weg} := (HHH, HEH, HHE, HEE, EHE, HEL, LHE, ELH, HLH, HHL, HLL, LHL, EEE, ELE, EEL, ELL, LEL, LLL) \\
p_{3-Kreis} := (HHH, HEH, HEE, HEL, LEH, HLH, HLL, EEE, ELE, ELL, LLL) \\
\end{align}
\end{subequations}

Die Vektoren $a_2, a_{3-Weg}$ und $a_{3-Kreis}$ beschreiben die Worte f\"ur \textit{Graphlets} in AA-Graphen 

\begin{subequations}
\begin{align}
a_2 := (hh, hp, ha, h?, pp, pa, p?, aa, a?, ??) \\
a_{3-Weg} := (hhh, hph, hhp, hpp, php, hpa, ahp, pah, hp?, ?hp, p?h, hah, hha, haa, aha, ha?, ?ha ,a?h, h?h, hh?, h??, ?h?, ppp, pap, ppa, paa, apa, pa?, ?pa, a?p, p?p, pp?, p??, ?p?, aaa, a?a, aa?, a??, ?a?, ???) \\
a_{3-Kreis} := (hhh, hph, hpp, hpa, aph, hp?, ?ph, hah, haa, ha?, ?ah, h?h, h??, ppp, pap, paa, pa?, ?ap, p?p, p??, aaa, a?a, a??, ???)
\end{align}
\end{subequations}


\section{Erweiterung von \texttt{graphletAnalyser}}

\paragraph{Das Einlesen von Komplexgraphen und Aminos\"auregraphen}

ist implementiert worden. Die entsprechenden Alphabete sind im Programmcode vordefiniert und k\"onnen vom Nutzer \"uber Parameter ausgew\"ahlt werden oder in der Konfigurationsdatei festgelegt werden.

\paragraph{Nutzerdefinierte Knotenmarkierungen} k\"onnen nun in der Konfigurationsdatei angegeben werden. Der Nutzer kann ein Alphabet von Knotenmarkierungen und ein \textit{Label}, unter dem diese Knotenmarkierungen in den GML-Dateien gespeichert sind, angeben. F\"ur dieses Alphabet werden alle \"aquivalenten \textit{Graphlet}-Worte durch den \textit{Graphlet}-Worte-Algorithmus berechnet. Diese werden dann bei der Berechnung der markierten \textit{Graphlets} im Graphen unter dem vorgegebenen \textit{Label} gesucht und gez\"ahlt.
Es k\"onnen also beliebige Alphabete und \textit{Labels} angegeben werden, so lange die Markierungen der Knoten nicht mehr als einen Buchstaben enthalten.


 

\paragraph{Die Datenbankanbindung} wurde um Funktionen zum Speichern von Aminos\"auregraphen und Komplexgraphen erweitert. Das Speichern von Vektoren markierter \textit{Graphlets} wurde implementiert.

Wenn die Option \texttt{--useDatabase} ausgew\"ahlt wird, pr\"uft das Programm, ob der entsprechende Graph bereits in der Datenbank vorhanden ist. Falls der Graph gefunden wurde, wird der \textit{Graphlet}-Vektor f\"ur den entsprechenden Graphen in die Datenbank eingetragen. 



\section{Fallstudien - Datensatz 1}

Die paarweisen RGF-Distanzen der Proteine befinden sich in den Tabellen \ref{table:occ-pg-rgf} \ref{table:occ-aag-rgf} und \ref{table:occ-cg-tc}. Die paarweisen Jaccard-Indizes befinden sich in den Tabellen \ref{table:occ-aa-tc}, \ref{table:occ-pg-tf} und \ref{table:occ-cg-tc}.
Hierbei wurden die Zellen, die die 4 besten Bewertungen f\"ur das Protein der entsprechenden Zeile enthalten gr\"un eingef\"arbt. Hierbei gilt, dass das Gr\"un umso dunkler ist, je st\"arker die \"Ahnlichkeit bewertet wird.

\paragraph{Der Vergleich mittels RGF}
zeigt f\"ur die Aminos\"auregraphen die st\"arksten \"Ahnlichkeiten immer innerhalb der entsprechenden CATH-Klassen. Sowohl die Proteine aus der \textit{mainly-alpha}-Klasse, als auch die Proteine der \textit{mainly-beta}-Klasse haben die besten \"Ahnlichkeitswerte mit Proteinen der gleichen Klasse.
Innerhalb der Klasse der \textit{alpha-beta}-Proteine, gibt es mit 2id9 und 1d4o zwei Proteine, denen eine gr\"o"sere \"Ahnlichkeit zu Proteinen der \textit{mainly-alpha}-Klasse attestiert wurde.
Bei 1d4o f\"allt auf, dass der niedrigste Wert mit 7,091 deutlich h\"oher ist, als die besten Werte aller anderen Proteine.
2id9 hat laut der RGF-Distanz die gr\"o"ste \"Ahnlichkeit zu 1he9.
Bis auf diese beiden Au"snahmen l\"asst sich jedoch eine starke Korrelation mit den CATH-Klassen erkennen.
Alle anderen Proteine haben mindestens die zwei kleinsten zwei RGF-Distanzen zu Vertretern aus der selben CATH-Klasse.

Dies gilt f\"ur die Aminos\"auregraphen. Die RGF-Distanzen der Proteingraphen zueinander zeigen ein weniger klares Bild.
F\"ur die \textit{mainly-alpha}-Klasse und die der \textit{mainly-beta}-Klasse befinden sich die Proteine mit den k\"urzesten Distanzen immer noch in der selben Klasse. Dies l\"asst sich f\"ur die Proteine der \textit{alpha-beta}-Klasse aber nicht mehr behaupten. Hier haben 2ID9, 3I42 und 2w0I die k\"urzesten Distanzen zu Proteinen anderer Klassen.

Bei den Komplexgraphen ist die Korrelation zwischen der RGF-Distanz und der Zugeh\"origkeit zur CATH-Klasse noch geringer. Die Tabelle \ref{table:occ-cg-rgf} zeigt nur f\"ur die Proteine 1QQ3, 1HE9, 1EXS und 1QQS die k\"urzeste Distanz zu einem Vertreter der gleichen Klasse.
Es f\"allt jedoch auf, dass besonders h\"aufig die Proteine der \textit{alpha-beta}-Klasse 5CHY, 2ID9 und 3I42 als \"ahnlich zu anderen bewertet werden.

\paragraph{Der Vergleich der Jaccard-Indizes}
zeigt ein \"ahnliches Bild, wie der Vergleich der RGF-Distanzen. Bei den Aminos\"auregraphen zeigt sich, dass innerhalb der \textit{mainly-alpha}-Klasse wieder die paarweisen \"Ahnlichkeiten der \textit{mainly-alpha}-Proteine am gr\"o"sten sind. Dies gilt bis auf eine Ausnahme auch f\"ur die \textit{mainly-beta}-Proteine. Das Protein mit der PDB-ID 1NGL wird als strukturell \"ahnlichstes Protein zu 2W0I bewertet. In der Klasse der \textit{alpha-beta}-Proteine gibt es mit 2ID9 und 1D4O wieder zwei Ausrei"ser, die die gr\"o"sten paarweisen \"Ahnlichkeiten nicht zu Vertretern der eigenen Klasse haben. F\"ur 2ID9 wird 1HE9 als \"ahnlichstes Protein angegeben und 1D4O wird 1QQS zugeordnet.

Die paarweisen Tanimoto-Koeffizienten der Proteingraphen zeigen - wie schon bei den RGF-Distanzen - eine geringere Korrelation mit der Zugeh\"origkeit zu den CATH-Klassen, als die Koeffizienten der Aminos\"auregraphen. Es haben zwar wieder mindestens 3 Vertreter jeder Klasse ihren n\"achsten Nachbarn in der gleichen Klasse, aber es gibt auch einige Proteine, die ihren n\"achsten Nachbarn au"serhalb der eigenen Klasse haben. Hierzu geh\"oren 1D4O, 2W0I (beide \textit{alpha-beta}) und 3GF9 (\textit{mainly-alpha}). Es f\"allt auf, dass wieder die Proteine mit den PDB-DIs 2ID9 und 3I42 besonders h\"aufig als strukturell \"ahnlich zu vielen anderen Proteinen bewertet werden.

F\"ur die Komplexgraphen zeigt die Tabelle wieder eine hohe \"Ahnlichkeite unter den ersten 3 Proteinen 1QPU, 1QQ3 und 1CGN. Auch innerhalb der Klasse \textit{alpha-beta} sind 3 Proteine mit der h\"ochsten paarweisen \"Ahnlichkeit bewertet worden. Die geringe Anzahl von stark bewerteten \"Ahnlichkeiten innerhalb der \textit{mainly-beta}-Klasse ist sehr auff\"allig. 1QQS und 3SLO sind das einzige Paar mit \textit{beta}-Topologie, dessen \"Ahnlichkeit als gro"s bewertet wurde.

\section{Fallstudien - Datensatz 2}

\section{PDBTop500-Datensatz}

F\"ur den PDBTop500-Datensatz wurden \textit{Graphlet}-Vektoren der drei verschiedenen Graphenformate berechnet. Diese Vektoren wurden f\"ur jedes Format in einer .csv-Datei zusammengefasst und f\"ur jede dieser Dateien wurden \textit{Boxplots} erstellt.

Mit dieser Darstellung kann ermittelt werden, f\"ur welche \textit{Graphlets} die Streuung besonders gro"s ist. Diese \textit{Graphlets} sollten dann besonders gut geeignet sein, um die entsprechenden Graphen zu klassifizieren.

Der folgende Plot zeigt die Verteilungen f\"ur die Aminos\"auregraphen.



\begin{figure}[h!]
\centering
\label{fig:aag_dist}
\includegraphics[scale=0.9]{graphlet_verteilungen_aag.png}
\caption{\textit{Graphlets} f\"ur die Aminos\"auregraphen des PDBTop500-Datensatzes. Jeder Eintrag auf der x-Achse entspricht einem \textit{Graphlet}. Die y-Achse stellt ihre relativen H\"aufigkeiten dar}
\end{figure}

In diesem Plot entsprechen die ersten 29 Eintr\"age den \textit{Graphlets} ohne Markierungen. F\"ur diese \textit{Graphlets} ist die Varianz der Relativen-\textit{Graphlet}-H\"aufigkeiten deutlich gr\"o"ser, als f\"ur die markierten \textit{Graphlets}. Damit haben diese auch den gr\"o"sten Einfluss auf die Vergleiche der \textit{Graphlet}-Vektoren.

\chapter{Diskussion und Ausblick}



\section{Diskussion}

Die folgende Diskussion der Fallstudien widmet sich vor allem der Frage, wieso die Ergebnisse der \"Ahnlichkeitsvergleiche, sich so stark zwischen den jeweiligen Graphendarstellungen unterscheiden.
Des weiteren wird der Zusammenhang zwischen dem Jaccard-Index und der RGF untersucht und es wird \"uberpr\"uft, ob das Z\"ahlen markierter \textit{Graphlets} zu einer h\"oheren Genauigkeit f\"uhrt.

\subsection{Datensatz 1}
\subsubsection{Vergleich der Graphformate}

Wie schon im Ergebnisteil dargestellt, zeigen die Vergleiche der Aminos\"auregraphen den h\"ohsten Konsens mit der Einteilung der Strukturen durch CATH und SCOPe. Eine m\"ogliche Erkl\"arung hierf\"ur ist die Gestalt der Graphen. Bisher wurden \textit{Graphlets} zur Analyse von zusammenh\"angenden Graphen verwendet (\cite{sherv_graphlets}, \cite{graphletfrequency}).
Proteingraphen und Komplexgraphen sind jedoch nicht immer zusammenh\"angend. es kommt h\"aufig vor, dass einzelne Knoten keine Verbindungen zum Rest des Graphen aufweisen. Das unten stehende Bild zeigt ein Beispiel.

\begin{figure}[h!]
\includegraphics[scale=0.5]{3gf9_A_albe_PG.png}
\caption{Proteingraph von 3GF9 - Datensatz 1}
\end{figure}

Dadurch, dass dies in den zusammenh\"angenden \textit{Graphlets} nicht ber\"ucksichtigt werden kann, geht Information verloren. Unabh\"angig von der Wahl des \"Ahnlichkeitsma"s w\"urde dieser Graph mit einem anderen Graphen, dem die beiden Helix-Knoten mit einem Grad von 0 fehlen, als gleich bewertet werden, obwohl dieser zwei SSEs weniger aufwiese. Diese SSEs k\"onnen jedoch biologisch von zentraler Bedeutung sein.

Im Gegensatz hierzu sind die Aminos\"auregraphen dieser Fallstudie zusammenh\"angend. Dies erkl\"art die h\"ohere Genauigkeit.

\subsubsection{Vergleich der Distanzma"se}

Wirft man einen Blick in die Tabellen, sieht es zun\"achst so aus, als w\"urden sich Jaccard-Index und RGF \"ahnlich gut eignen, um die \"Ahnlichkeit der \textit{Graphlet}-Vektoren zu bewerten. Die RGF stellt eine \emph{Distanz} zwischen zwei Vektoren dar. Dementsprechend steht ein RGF-Wert von 0 f\"ur die Gleichheit zweier Vektoren, je h\"oher der Wert ist, desto h\"oher ist der Abstand zwischen den beiden Vektoren. Der modifizierte Jaccard-Index, der hier Verwendung findet, z\"ahlt Elemente, die sich um h\"ochstens einen Faktor $k$ unterscheiden. Ein RGF-Wert von 1 bedeutete, dass alle Elemente beider Vektoren sich h\"ochstens um den Faktor $k$ unterscheiden. Ein Wert von 0 bedeutet, dass alle Elemente sich um mehr als den Faktor $k$ unterscheiden. Dementsprechend w\"urde man erwarten, dass die \textit{Pearson}-Korrelation von RGF und Jaccard-Index negativ ist. Die folgende Tabelle zeigt jedoch, dass diese f\"ur alle berechneten Daten positiv ausf\"allt. 


\begin{table}
\begin{tabular}{c c c}

Vektor & RGF & Jaccard-Index \\

\end{tabular}
\end{table}


\subsection{Datensatz 2}

\subsection{PDBTop500-Datensatz}

\paragraph{Aminos\"auregraphen} scheinen wenig geeignet f\"ur die Anwendung von markierten \textit{Graphlets} zu sein, da die meisten von ihnen garnicht gefunden werden, oder ihre relative H\"aufigkeit so klein ist, dass sie kaum einen Einfluss auf den Vergleich der \textit{Graphlet}-Vektoren hat.

Dies kommt daher, dass Aminos\"auregraphen zum einen deutlich mehr Knoten und Kanten enthalten als Komplexgraphen und Proteingraphen und f\"ur ihre 3-\textit{Graphlets} deutlich mehr repr\"asentierende Worte existieren.

 


TODO: durchschnittliche Knoten, Kantenzahl, Knotengrade der unterschiedlichen Foramte berechnen 



\section{Ausblick}


\subsection{Optimierung der Laufzeit von \texttt{graphletAnalyser}}

Vor allem bei der Berechnung der \textit{Graphlets} auf Aminos\"aurgraphen gro"ser Proteine besteht Verbesserungspotential. Aktuell z\"ahlt das Programm bei einem Aufruf automatische alle \textit{Graphlets} mit 3,4 und 5 Knoten jeweils getrennt voneinander.

Da f\"ur die \"Ahnlichkeitsberechnung in ihrer aktuellen Form alle \textit{Graphlets} einbezogen werden, ist es sinnvoll diese Berechnungen in einem Algorithmus zusammenzufassen. F\"ur Graphen mit $n$ Knoten und einem maximalen Knotengrad $d$ lie"se sich die Laufzeit damit von $O(nd^4+nd^3+nd^2)$ auf $O(nd^4)$ reduzieren.

Diese Zusammenfassung lie"se sich dadurch bewerkstelligen, dass man die Funktionen zum \"Uberpr\"ufen der \textit{Graphlets} aus den Algorithmen zum Z\"ahlen der 3- und 4-\textit{Graphlets} in den Algorithmus zum Z\"ahlen der 5-\textit{Graphlets} in die \textit{for}-Schleifen der entsprechenden Tiefen einf\"ugt.

\subsection{\"Ahnlichkeitssuche}

\subsection{\textit{Graphlet}-Motive}

\subsection{Klassifizierung mit \textit{Support-Vector-Machines}}




\chapter{Anhang}


\section{Bildverzeichnis}

\begin{figure}[h!]
\includegraphics[width =\linewidth]{3graphlets.pdf}
\caption{Graphlets der Gr\"o"se 3 (\textit{Shervashidze et al.})}
\label{fig:3graphlets}
\end{figure}

\begin{figure}[h!]
\includegraphics[width =\linewidth]{4graphlets.pdf}
\caption{Graphlets der Gr\"o"se 4 (\textit{Shervashidze et al.})}
\label{fig:4graphlets}
\end{figure}

\newpage

\begin{figure}[h!]
\includegraphics[width =\linewidth]{5graphlets.pdf}
\caption{Graphlets der Gr\"o"se 5 (\textit{Shervashidze et al.})}
\label{fig:5graphlets}
\end{figure}

\newpage

%TODO: UML Diagramm so, konvertieren, dass es eine Bounding Box hat und wieder einf\"ugen

\section{Tabellenverzeichnis}

\newpage


\begin{sidewaystable}

\definecolor{fGreen}{rgb}{0.13,0.54,0.13}

\scalebox{0.8}{
\begin{tabular}[h!]{l l l l l l l l l l l l l l l l l l l l l l l l}

PDB-ID & \cellcolor{fGreen!100}2utg & 1vib & 1qpu & 1qq3 & 2xl6 & 2yl1 & 1rxq & 2qe9 & 2rd9 & 1m4r & 1hgu & 1pv6 & 1awr & 1d9c & 1ngl & 1n7v & 7tim & 1ney & 1v3z & 1a17 & 1ar0 & 2v5l &  \\
2utg &   X   & 8.915 & 8.626 & 8.150 & 9.759 & 13.62 & 6.408 & 8.128 & 7.384 & 5.158 & 4.944 & 3.570 & 11.67 & 4.392 & 7.387 & 13.02 & 9.171 & 8.737 & 8.025 & 2.490 & 9.329 & 6.253 &  \\
1vib & 8.915 &   X   & 15.54 & 15.89 & 16.89 & 21.08 & 15.22 & 16.80 & 16.23 & 13.39 & 13.26 & 12.32 & 17.13 & 12.55 & 11.80 & 17.87 & 17.44 & 17.07 & 13.78 & 7.943 & 16.33 & 14.27 &  \\
1qpu & 8.626 & 15.54 &   X   & 1.451 & 3.655 & 5.632 & 6.272 & 8.187 & 7.539 & 5.410 & 7.849 & 7.087 & 11.01 & 7.053 & 9.864 & 12.54 & 8.261 & 8.829 & 8.822 & 9.192 & 9.692 & 6.462 &  \\
1qq3 & 8.150 & 15.89 & 1.451 &   X   & 3.614 & 5.651 & 6.064 & 7.554 & 7.252 & 5.087 & 7.413 & 6.776 & 11.02 & 6.721 & 9.918 & 12.56 & 7.787 & 8.354 & 8.892 & 9.531 & 9.216 & 6.019 &  \\
2xl6 & 9.759 & 16.89 & 3.655 & 3.614 &   X   & 4.518 & 5.106 & 5.523 & 5.482 & 5.514 & 8.147 & 8.655 & 8.374 & 6.685 & 8.091 & 9.919 & 5.950 & 6.459 & 6.255 & 10.30 & 7.054 & 4.716 &  \\
2yl1 & 13.62 & 21.08 & 5.632 & 5.651 & 4.518 &   X   & 7.737 & 7.197 & 7.589 & 8.855 & 11.36 & 11.82 & 11.53 & 10.34 & 12.29 & 12.94 & 8.343 & 8.923 & 10.59 & 14.75 & 10.14 & 8.137 &  \\
1rxq & 6.408 & 15.22 & 6.272 & 6.064 & 5.106 & 7.737 &   X   & 2.271 & 1.563 & 1.846 & 4.457 & 4.428 & 8.105 & 2.685 & 6.737 & 10.11 & 4.059 & 4.769 & 5.299 & 8.066 & 5.910 & 3.043 &  \\
2qe9 & 8.128 & 16.80 & 8.187 & 7.554 & 5.523 & 7.197 & 2.271 &   X   & 2.024 & 4.006 & 6.167 & 5.959 & 8.708 & 4.649 & 8.068 & 10.83 & 4.470 & 5.180 & 6.065 & 9.575 & 6.728 & 4.542 &  \\
2rd9 & 7.384 & 16.23 & 7.539 & 7.252 & 5.482 & 7.589 & 1.563 & 2.024 &   X   & 3.205 & 5.325 & 5.829 & 7.137 & 3.853 & 6.677 & 9.126 & 3.049 & 3.785 & 4.747 & 9.046 & 5.086 & 3.123 &  \\
1m4r & 5.158 & 13.39 & 5.410 & 5.087 & 5.514 & 8.855 & 1.846 & 4.006 & 3.205 &   X   & 3.657 & 3.784 & 8.468 & 1.837 & 6.243 & 10.41 & 5.063 & 5.650 & 5.068 & 6.336 & 5.972 & 2.922 &  \\
1hgu & 4.944 & 13.26 & 7.849 & 7.413 & 8.147 & 11.36 & 4.457 & 6.167 & 5.325 & 3.657 &   X   & 4.692 & 9.337 & 3.187 & 5.776 & 10.59 & 6.059 & 5.603 & 6.003 & 7.045 & 6.636 & 3.742 &  \\
1pv6 & 3.570 & 12.32 & 7.087 & 6.776 & 8.655 & 11.82 & 4.428 & 5.959 & 5.829 & 3.784 & 4.692 &   X   & 12.09 & 4.568 & 8.081 & 13.67 & 7.678 & 8.304 & 8.510 & 4.926 & 9.202 & 5.753 &  \\
1awr & 11.67 & 17.13 & 11.01 & 11.02 & 8.374 & 11.53 & 8.105 & 8.708 & 7.137 & 8.468 & 9.337 & 12.09 &   X   & 7.762 & 7.577 & 2.196 & 4.751 & 4.070 & 3.972 & 11.87 & 3.395 & 6.506 &  \\
1d9c & 4.392 & 12.55 & 7.053 & 6.721 & 6.685 & 10.34 & 2.685 & 4.649 & 3.853 & 1.837 & 3.187 & 4.568 & 7.762 &   X   & 6.366 & 9.743 & 5.605 & 5.246 & 4.468 & 5.626 & 6.116 & 3.637 &  \\
1ngl & 7.387 & 11.80 & 9.864 & 9.918 & 8.091 & 12.29 & 6.737 & 8.068 & 6.677 & 6.243 & 5.776 & 8.081 & 7.577 & 6.366 &   X   & 8.064 & 6.805 & 6.711 & 4.658 & 8.473 & 5.325 & 4.582 &  \\
1n7v & 13.02 & 17.87 & 12.54 & 12.56 & 9.919 & 12.94 & 10.11 & 10.83 & 9.126 & 10.41 & 10.59 & 13.67 & 2.196 & 9.743 & 8.064 &   X   & 6.405 & 5.675 & 5.484 & 13.53 & 4.509 & 7.987 &  \\
7tim & 9.171 & 17.44 & 8.261 & 7.787 & 5.950 & 8.343 & 4.059 & 4.470 & 3.049 & 5.063 & 6.059 & 7.678 & 4.751 & 5.605 & 6.805 & 6.405 &   X   & 0.919 & 3.869 & 11.02 & 2.858 & 3.619 &  \\
1ney & 8.737 & 17.07 & 8.829 & 8.354 & 6.459 & 8.923 & 4.769 & 5.180 & 3.785 & 5.650 & 5.603 & 8.304 & 4.070 & 5.246 & 6.711 & 5.675 & 0.919 &   X   & 3.500 & 10.59 & 2.683 & 3.646 &  \\
1v3z & 8.025 & 13.78 & 8.822 & 8.892 & 6.255 & 10.59 & 5.299 & 6.065 & 4.747 & 5.068 & 6.003 & 8.510 & 3.972 & 4.468 & 4.658 & 5.484 & 3.869 & 3.500 &   X   & 8.286 & 2.693 & 3.435 &  \\
1a17 & 2.490 & 7.943 & 9.192 & 9.531 & 10.30 & 14.75 & 8.066 & 9.575 & 9.046 & 6.336 & 7.045 & 4.926 & 11.87 & 5.626 & 8.473 & 13.53 & 11.02 & 10.59 & 8.286 &   X   & 10.56 & 7.612 &  \\
1ar0 & 9.329 & 16.33 & 9.692 & 9.216 & 7.054 & 10.14 & 5.910 & 6.728 & 5.086 & 5.972 & 6.636 & 9.202 & 3.395 & 6.116 & 5.325 & 4.509 & 2.858 & 2.683 & 2.693 & 10.56 &   X   & 3.679 &  \\
2v5l & 6.253 & 14.27 & 6.462 & 6.019 & 4.716 & 8.137 & 3.043 & 4.542 & 3.123 & 2.922 & 3.742 & 5.753 & 6.506 & 3.637 & 4.582 & 7.987 & 3.619 & 3.646 & 3.435 & 7.612 & 3.679 &   X   &  \\


\end{tabular}}

\end{sidewaystable}
\newpage




\begin{table}
\label{table:occ-aag-rgf}
\scalebox{0.8}{
\begin{tabular}{l l l l l l l l l l l l l l l l l}

PDB-ID & 1qpu & 1qq3 & 1cgn & 1he9 & 3gf9 & 1exs & 1ngl & 1qqs & 3slo & 1wjx & 5chy & 2id9 & 3i42 & 1d4o & 2w0i &  \\
1qpu &   X   & \cellcolor{fGreen!100}1.139 & \cellcolor{fGreen!75}1.296 & 6.159 & \cellcolor{fGreen!50}5.656 & 10.52 & 11.00 & 12.35 & 12.45 & 11.66 & 7.517 & \cellcolor{fGreen!25}6.128 & 6.625 & 7.326 & 8.572 &  \\
1qq3 & \cellcolor{fGreen!100}1.139 &   X   & \cellcolor{fGreen!75}1.310 & 5.980 & \cellcolor{fGreen!50}5.645 & 9.964 & 10.38 & 11.61 & 11.84 & 11.10 & 6.967 & \cellcolor{fGreen!25}5.776 & 6.039 & 7.491 & 8.059 &  \\
1cgn & \cellcolor{fGreen!100}1.296 & \cellcolor{fGreen!75}1.310 &   X   & 6.093 & \cellcolor{fGreen!25}5.737 & 9.987 & 10.44 & 11.55 & 11.81 & 11.16 & 6.801 & \cellcolor{fGreen!50}5.663 & 5.929 & 7.091 & 7.918 &  \\
1he9 & 6.159 & 5.980 & 6.093 &   X   & \cellcolor{fGreen!100}2.289 & 8.032 & 6.813 & 11.03 & 10.87 & 8.405 & \cellcolor{fGreen!25}3.974 & \cellcolor{fGreen!75}2.445 & \cellcolor{fGreen!50}3.452 & 10.93 & 4.807 &  \\
3gf9 & 5.656 & 5.645 & 5.737 & \cellcolor{fGreen!100}2.289 &   X   & 9.544 & 8.479 & 12.36 & 12.19 & 10.03 & \cellcolor{fGreen!25}5.135 & \cellcolor{fGreen!75}3.370 & \cellcolor{fGreen!50}4.359 & 12.08 & 6.018 &  \\
1exs & 10.52 & 9.964 & 9.987 & 8.032 & 9.544 &   X   & \cellcolor{fGreen!25}4.463 & \cellcolor{fGreen!50}4.041 & \cellcolor{fGreen!75}3.138 & \cellcolor{fGreen!100}2.802 & 4.773 & 7.086 & 5.541 & 7.374 & 4.530 &  \\
1ngl & 11.00 & 10.38 & 10.44 & 6.813 & 8.479 & \cellcolor{fGreen!50}4.463 &   X   & 7.716 & 6.865 & \cellcolor{fGreen!100}3.764 & \cellcolor{fGreen!25}4.718 & 6.054 & 5.065 & 9.755 & \cellcolor{fGreen!75}4.031 &  \\
1qqs & 12.35 & 11.61 & 11.55 & 11.03 & 12.36 & \cellcolor{fGreen!75}4.041 & 7.716 &   X   & \cellcolor{fGreen!100}1.674 & \cellcolor{fGreen!50}5.340 & 8.148 & 10.20 & 8.612 & \cellcolor{fGreen!25}7.226 & 7.404 &  \\
3slo & 12.45 & 11.84 & 11.81 & 10.87 & 12.19 & \cellcolor{fGreen!75}3.138 & \cellcolor{fGreen!25}6.865 & \cellcolor{fGreen!100}1.674 &   X   & \cellcolor{fGreen!50}4.750 & 7.488 & 9.994 & 8.600 & 7.680 & 6.997 &  \\
1wjx & 11.66 & 11.10 & 11.16 & 8.405 & 10.03 & \cellcolor{fGreen!100}2.802 & \cellcolor{fGreen!75}3.764 & 5.340 & \cellcolor{fGreen!25}4.750 &   X   & 5.264 & 7.522 & 5.699 & 8.450 & \cellcolor{fGreen!50}4.384 &  \\
5chy & 7.517 & 6.967 & 6.801 & \cellcolor{fGreen!25}3.974 & 5.135 & 4.773 & 4.718 & 8.148 & 7.488 & 5.264 &   X   & \cellcolor{fGreen!75}2.600 & \cellcolor{fGreen!50}2.817 & 8.667 & \cellcolor{fGreen!100}1.497 &  \\
2id9 & 6.128 & 5.776 & 5.663 & \cellcolor{fGreen!100}2.445 & \cellcolor{fGreen!25}3.370 & 7.086 & 6.054 & 10.20 & 9.994 & 7.522 & \cellcolor{fGreen!50}2.600 &   X   & \cellcolor{fGreen!75}2.447 & 10.24 & 3.657 &  \\
3i42 & 6.625 & 6.039 & 5.929 & \cellcolor{fGreen!25}3.452 & 4.359 & 5.541 & 5.065 & 8.612 & 8.600 & 5.699 & \cellcolor{fGreen!75}2.817 & \cellcolor{fGreen!100}2.447 &   X   & 9.544 & \cellcolor{fGreen!50}2.817 &  \\
1d4o & \cellcolor{fGreen!50}7.326 & 7.491 & \cellcolor{fGreen!100}7.091 & 10.93 & 12.08 & \cellcolor{fGreen!25}7.374 & 9.755 & \cellcolor{fGreen!75}7.226 & 7.680 & 8.450 & 8.667 & 10.24 & 9.544 &   X   & 8.970 &  \\
2w0i & 8.572 & 8.059 & 7.918 & 4.807 & 6.018 & 4.530 & \cellcolor{fGreen!25}4.031 & 7.404 & 6.997 & 4.384 & \cellcolor{fGreen!100}1.497 & \cellcolor{fGreen!50}3.657 & \cellcolor{fGreen!75}2.817 & 8.970 &   X   &  \\



\end{tabular}}
\caption{Distanzen der Aminos\"auregraphen, gemessen mit RGF. In den Zellen der Tabelle stehen die RGF-Distanzen f\"ur die entsprechenden PDB-Dateien. F\"ur jede Zeile (jedes Protein) sind die 4 niedrigsten Distanzen gr\"un unterlegt. Je dunkler das gr\"un ist, desto k\"urzer ist die Distanz}
\end{table}



\begin{table}
\label{table:occ-pg-rgf}
\scalebox{0.8}{
\begin{tabular}{l l l l l l l l l l l l l l l l l}


PDB-ID & 1qpu & 1qq3 & 1cgn & 1he9 & 3gf9 & 1exs & 1ngl & 1qqs & 3slo & 1wjx & 5chy & 2id9 & 3i42 & 1d4o & 2w0i &  \\
1qpu &   X   & \cellcolor{fGreen!25}0.806 & \cellcolor{fGreen!100}0.606 & \cellcolor{fGreen!50}0.697 & 3.131 & 1.159 & 2.195 & 2.197 & 1.226 & 1.049 & 1.014 & 0.811 & 1.098 & \cellcolor{fGreen!75}0.628 & 0.972 &  \\
1qq3 & \cellcolor{fGreen!25}0.806 &   X   & \cellcolor{fGreen!100}0.405 & 1.504 & 1.567 & 1.703 & 0.883 & 0.883 & 1.396 & 1.270 & 0.869 & \cellcolor{fGreen!75}0.405 & \cellcolor{fGreen!50}0.693 & 0.988 & 1.779 &  \\
1cgn & \cellcolor{fGreen!75}0.606 & \cellcolor{fGreen!100}0.405 &   X   & \cellcolor{fGreen!50}0.810 & 1.432 & 1.633 & 1.193 & 1.193 & 1.513 & 1.339 & 1.060 & \cellcolor{fGreen!25}0.811 & 1.098 & 1.011 & 1.516 &  \\
1he9 & \cellcolor{fGreen!100}0.697 & 1.504 & \cellcolor{fGreen!75}0.810 &   X   & 4.492 & 3.253 & 1.386 & 1.386 & 4.357 & 1.291 & \cellcolor{fGreen!25}1.135 & \cellcolor{fGreen!50}1.099 & 1.386 & 3.599 & 2.506 &  \\
3gf9 & 3.131 & \cellcolor{fGreen!75}1.567 & \cellcolor{fGreen!100}1.432 & 4.492 &   X   & 2.523 & 2.815 & 2.817 & \cellcolor{fGreen!50}1.596 & 2.056 & \cellcolor{fGreen!25}1.816 & 1.917 & 2.034 & 3.122 & 4.130 &  \\
1exs & \cellcolor{fGreen!50}1.159 & 1.703 & 1.633 & 3.253 & 2.523 &   X   & \cellcolor{fGreen!25}1.324 & 1.324 & 2.968 & \cellcolor{fGreen!100}0.696 & \cellcolor{fGreen!75}0.885 & 1.492 & 1.610 & 3.102 & 3.820 &  \\
1ngl & 2.195 & 0.883 & 1.193 & 1.386 & 2.815 & 1.324 &   X   & \cellcolor{fGreen!100}0.002 & \cellcolor{fGreen!50}0.787 & 2.629 & 2.279 & \cellcolor{fGreen!25}0.788 & \cellcolor{fGreen!75}0.286 & 0.980 & 2.722 &  \\
1qqs & 2.197 & 0.883 & 1.193 & 1.386 & 2.817 & 1.324 & \cellcolor{fGreen!100}0.002 &   X   & \cellcolor{fGreen!50}0.787 & 2.631 & 2.281 & \cellcolor{fGreen!25}0.788 & \cellcolor{fGreen!75}0.285 & 0.980 & 2.722 &  \\
3slo & 1.226 & 1.396 & 1.513 & 4.357 & 1.596 & 2.968 & \cellcolor{fGreen!50}0.787 & \cellcolor{fGreen!25}0.787 &   X   & \cellcolor{fGreen!100}0.257 & \cellcolor{fGreen!75}0.705 & 1.024 & 1.073 & 3.137 & 6.154 &  \\
1wjx & 1.049 & 1.270 & 1.339 & 1.291 & 2.056 & \cellcolor{fGreen!50}0.696 & 2.629 & 2.631 & \cellcolor{fGreen!100}0.257 &   X   & 1.043 & 0.933 & \cellcolor{fGreen!25}0.914 & \cellcolor{fGreen!75}0.644 & 2.093 &  \\
5chy & 1.014 & 0.869 & 1.060 & 1.135 & 1.816 & 0.885 & 2.279 & 2.281 & \cellcolor{fGreen!50}0.705 & 1.043 &   X   & \cellcolor{fGreen!75}0.607 & \cellcolor{fGreen!25}0.800 & \cellcolor{fGreen!100}0.275 & 2.282 &  \\
2id9 & 0.811 & \cellcolor{fGreen!100}0.405 & 0.811 & 1.099 & 1.917 & 1.492 & 0.788 & 0.788 & 1.024 & 0.933 & \cellcolor{fGreen!75}0.607 &   X   & \cellcolor{fGreen!50}0.692 & \cellcolor{fGreen!25}0.783 & 2.890 &  \\
3i42 & 1.098 & \cellcolor{fGreen!25}0.693 & 1.098 & 1.386 & 2.034 & 1.610 & \cellcolor{fGreen!75}0.286 & \cellcolor{fGreen!100}0.285 & 1.073 & 0.914 & 0.800 & \cellcolor{fGreen!50}0.692 &   X   & 1.076 & 3.008 &  \\
1d4o & \cellcolor{fGreen!75}0.628 & 0.988 & 1.011 & 3.599 & 3.122 & 3.102 & 0.980 & 0.980 & 3.137 & \cellcolor{fGreen!50}0.644 & \cellcolor{fGreen!100}0.275 & \cellcolor{fGreen!25}0.783 & 1.076 &   X   & 4.406 &  \\
2w0i & \cellcolor{fGreen!100}0.972 & \cellcolor{fGreen!50}1.779 & \cellcolor{fGreen!75}1.516 & 2.506 & 4.130 & 3.820 & 2.722 & 2.722 & 6.154 & \cellcolor{fGreen!25}2.093 & 2.282 & 2.890 & 3.008 & 4.406 &   X   &  \\


\end{tabular}}
\caption{Distanzen der Proteingraphen, gemessen mit RGF. In den Zellen der Tabelle stehen die RGF-Distanzen f\"ur die entsprechenden PDB-Dateien. F\"ur jede Zeile (jedes Protein) sind die 4 niedrigsten Distanzen gr\"un unterlegt. Je dunkler das gr\"un ist, desto k\"urzer ist die Distanz}
\end{table}


\begin{table}
\label{table:occ-cg-rgf}
\scalebox{0.8}{
\begin{tabular}{l l l l l l l l l l l l l l l l l}

PDB-ID & 1qpu & 1qq3 & 1cgn & 1he9 & 3gf9 & 1exs & 1ngl & 1qqs & 3slo & 1wjx & 5chy & 2id9 & 3i42 & 1d4o & 2w0i &  \\
1qpu &   X   & 4.405 & 5.056 & 2.271 & 10.75 & 3.567 & 2.309 & 13.66 & 14.10 & \cellcolor{fGreen!50}2.128 & \cellcolor{fGreen!75}1.933 & \cellcolor{fGreen!25}2.165 & \cellcolor{fGreen!100}1.877 & 8.791 & 4.422 &  \\
1qq3 & 4.405 &   X   & \cellcolor{fGreen!75}1.906 & \cellcolor{fGreen!100}1.714 & 10.03 & 3.860 & 3.457 & 6.851 & 12.70 & 2.587 & \cellcolor{fGreen!25}2.431 & 2.557 & \cellcolor{fGreen!50}2.269 & 7.907 & 3.002 &  \\
1cgn & 5.056 & 1.906 &   X   & \cellcolor{fGreen!25}0.985 & 9.469 & 1.712 & 1.979 & 7.333 & 12.20 & 1.954 & 1.120 & \cellcolor{fGreen!75}0.249 & \cellcolor{fGreen!100}0.037 & 6.501 & \cellcolor{fGreen!50}0.980 &  \\
1he9 & 2.271 & 1.714 & \cellcolor{fGreen!100}0.985 &   X   & 5.881 & 3.456 & \cellcolor{fGreen!25}1.386 & 4.069 & 5.432 & 1.466 & \cellcolor{fGreen!50}1.291 & \cellcolor{fGreen!75}1.099 & 1.386 & 9.950 & 2.506 &  \\
3gf9 & 10.75 & 10.03 & 9.469 & 5.881 &   X   & 5.473 & 4.230 & 14.86 & 15.99 & \cellcolor{fGreen!75}2.516 & \cellcolor{fGreen!100}1.971 & \cellcolor{fGreen!50}2.575 & \cellcolor{fGreen!25}2.693 & 8.811 & 11.61 &  \\
1exs & 3.567 & 3.860 & 1.712 & 3.456 & 5.473 &   X   & \cellcolor{fGreen!50}1.290 & 3.217 & 4.391 & \cellcolor{fGreen!100}0.611 & \cellcolor{fGreen!75}0.662 & \cellcolor{fGreen!25}1.459 & 1.576 & 9.191 & 4.226 &  \\
1ngl & 2.309 & 3.457 & 1.979 & \cellcolor{fGreen!25}1.386 & 4.230 & \cellcolor{fGreen!50}1.290 &   X   & 3.637 & 2.620 & 2.697 & 2.644 & \cellcolor{fGreen!75}0.788 & \cellcolor{fGreen!100}0.286 & 4.373 & 2.722 &  \\
1qqs & 13.66 & 6.851 & 7.333 & 4.069 & 14.86 & \cellcolor{fGreen!100}3.217 & \cellcolor{fGreen!25}3.637 &   X   & 12.90 & 5.645 & 4.202 & \cellcolor{fGreen!75}3.394 & \cellcolor{fGreen!50}3.512 & 16.22 & 4.907 &  \\
3slo & 14.10 & 12.70 & 12.20 & 5.432 & 15.99 & 4.391 & \cellcolor{fGreen!50}2.620 & 12.90 &   X   & 4.974 & \cellcolor{fGreen!25}3.863 & \cellcolor{fGreen!100}2.351 & \cellcolor{fGreen!75}2.412 & 14.09 & 7.116 &  \\
1wjx & 2.128 & 2.587 & 1.954 & 1.466 & 2.516 & \cellcolor{fGreen!75}0.611 & 2.697 & 5.645 & 4.974 &   X   & \cellcolor{fGreen!100}0.360 & \cellcolor{fGreen!50}0.961 & \cellcolor{fGreen!25}0.965 & 2.225 & 2.042 &  \\
5chy & 1.933 & 2.431 & 1.120 & 1.291 & 1.971 & \cellcolor{fGreen!75}0.662 & 2.644 & 4.202 & 3.863 & \cellcolor{fGreen!100}0.360 &   X   & \cellcolor{fGreen!50}0.797 & \cellcolor{fGreen!25}0.914 & 1.728 & 2.093 &  \\
2id9 & 2.165 & 2.557 & \cellcolor{fGreen!100}0.249 & 1.099 & 2.575 & 1.459 & \cellcolor{fGreen!50}0.788 & 3.394 & 2.351 & 0.961 & \cellcolor{fGreen!25}0.797 &   X   & \cellcolor{fGreen!75}0.692 & 1.921 & 2.890 &  \\
3i42 & 1.877 & 2.269 & \cellcolor{fGreen!100}0.037 & 1.386 & 2.693 & 1.576 & \cellcolor{fGreen!75}0.286 & 3.512 & 2.412 & 0.965 & \cellcolor{fGreen!25}0.914 & \cellcolor{fGreen!50}0.692 &   X   & 2.039 & 3.008 &  \\
1d4o & 8.791 & 7.907 & 6.501 & 9.950 & 8.811 & 9.191 & 4.373 & 16.22 & 14.09 & \cellcolor{fGreen!25}2.225 & \cellcolor{fGreen!100}1.728 & \cellcolor{fGreen!75}1.921 & \cellcolor{fGreen!50}2.039 &   X   & 10.76 &  \\
2w0i & 4.422 & 3.002 & \cellcolor{fGreen!100}0.980 & \cellcolor{fGreen!25}2.506 & 11.61 & 4.226 & 2.722 & 4.907 & 7.116 & \cellcolor{fGreen!75}2.042 & \cellcolor{fGreen!50}2.093 & 2.890 & 3.008 & 10.76 &   X   &  \\



\end{tabular}}
\caption{Distanzen der Komplexgraphen, gemessen mit RGF. In den Zellen der Tabelle stehen die RGF-Distanzen f\"ur die entsprechenden PDB-Dateien. F\"ur jede Zeile (jedes Protein) sind die 4 niedrigsten Distanzen gr\"un unterlegt. Je dunkler das gr\"un ist, desto k\"urzer ist die Distanz}
\end{table}



\newpage

\begin{table}
\label{table:occ-aa-tc}
\scalebox{0.8}{
\begin{tabular}{l l l l l l l l l l l l l l l l l}

PDB-ID & 1qpu & 1qq3 & 1cgn & 1he9 & 3gf9 & 1exs & 1ngl & 1qqs & 3slo & 1wjx & 5chy & 2id9 & 3i42 & 1d4o & 2w0i &  \\
1qpu &   X   & \cellcolor{fGreen!75}1.0 & \cellcolor{fGreen!100}1.0 & 0.621 & \cellcolor{fGreen!25}0.666 & 0.276 & 0.25 & 0.25 & 0.224 & 0.25 & 0.538 & \cellcolor{fGreen!50}0.714 & 0.5 & 0.363 & 0.333 &  \\
1qq3 & \cellcolor{fGreen!75}1.0 &   X   & \cellcolor{fGreen!100}1.0 & 0.621 & 0.621 & 0.304 & 0.276 & 0.304 & 0.224 & 0.25 & 0.578 & \cellcolor{fGreen!50}0.666 & \cellcolor{fGreen!25}0.666 & 0.428 & 0.463 &  \\
1cgn & \cellcolor{fGreen!100}1.0 & \cellcolor{fGreen!75}1.0 &   X   & 0.621 & \cellcolor{fGreen!50}0.714 & 0.333 & 0.224 & 0.224 & 0.224 & 0.276 & 0.5 & \cellcolor{fGreen!25}0.666 & 0.578 & 0.333 & 0.428 &  \\
1he9 & 0.621 & 0.621 & 0.621 &   X   & \cellcolor{fGreen!100}0.935 & 0.463 & 0.428 & 0.176 & 0.276 & 0.304 & \cellcolor{fGreen!25}0.621 & \cellcolor{fGreen!75}0.875 & \cellcolor{fGreen!50}0.764 & 0.333 & 0.463 &  \\
3gf9 & \cellcolor{fGreen!25}0.666 & 0.621 & \cellcolor{fGreen!50}0.714 & \cellcolor{fGreen!100}0.935 &   X   & 0.276 & 0.333 & 0.2 & 0.2 & 0.224 & 0.621 & \cellcolor{fGreen!75}0.764 & 0.621 & 0.25 & 0.463 &  \\
1exs & 0.276 & 0.304 & 0.333 & 0.463 & 0.276 &   X   & 0.621 & 0.621 & \cellcolor{fGreen!50}0.764 & \cellcolor{fGreen!100}0.875 & \cellcolor{fGreen!25}0.666 & 0.5 & 0.578 & 0.463 & \cellcolor{fGreen!75}0.818 &  \\
1ngl & 0.25 & 0.276 & 0.224 & 0.428 & 0.333 & \cellcolor{fGreen!50}0.621 &   X   & 0.276 & 0.428 & \cellcolor{fGreen!75}0.666 & \cellcolor{fGreen!25}0.621 & 0.538 & 0.578 & 0.463 & \cellcolor{fGreen!100}0.666 &  \\
1qqs & 0.25 & 0.304 & 0.224 & 0.176 & 0.2 & \cellcolor{fGreen!50}0.621 & 0.276 &   X   & \cellcolor{fGreen!100}1.0 & \cellcolor{fGreen!75}0.621 & \cellcolor{fGreen!25}0.5 & 0.276 & 0.395 & 0.463 & 0.463 &  \\
3slo & 0.224 & 0.224 & 0.224 & 0.276 & 0.2 & \cellcolor{fGreen!75}0.764 & 0.428 & \cellcolor{fGreen!100}1.0 &   X   & \cellcolor{fGreen!50}0.538 & 0.463 & 0.333 & 0.395 & 0.463 & \cellcolor{fGreen!25}0.5 &  \\
1wjx & 0.25 & 0.25 & 0.276 & 0.304 & 0.224 & \cellcolor{fGreen!100}0.875 & \cellcolor{fGreen!75}0.666 & \cellcolor{fGreen!50}0.621 & 0.538 &   X   & 0.538 & 0.363 & 0.5 & 0.395 & \cellcolor{fGreen!25}0.621 &  \\
5chy & 0.538 & 0.578 & 0.5 & 0.621 & 0.621 & \cellcolor{fGreen!25}0.666 & 0.621 & 0.5 & 0.463 & 0.538 &   X   & \cellcolor{fGreen!75}0.818 & \cellcolor{fGreen!50}0.764 & 0.428 & \cellcolor{fGreen!100}1.0 &  \\
2id9 & 0.714 & 0.666 & 0.666 & \cellcolor{fGreen!100}0.875 & \cellcolor{fGreen!25}0.764 & 0.5 & 0.538 & 0.276 & 0.333 & 0.363 & \cellcolor{fGreen!75}0.818 &   X   & \cellcolor{fGreen!50}0.818 & 0.304 & 0.621 &  \\
3i42 & 0.5 & 0.666 & 0.578 & \cellcolor{fGreen!50}0.764 & 0.621 & 0.578 & 0.578 & 0.395 & 0.395 & 0.5 & \cellcolor{fGreen!75}0.764 & \cellcolor{fGreen!100}0.818 &   X   & 0.333 & \cellcolor{fGreen!25}0.714 &  \\
1d4o & 0.363 & 0.428 & 0.333 & 0.333 & 0.25 & \cellcolor{fGreen!25}0.463 & \cellcolor{fGreen!50}0.463 & \cellcolor{fGreen!100}0.463 & \cellcolor{fGreen!75}0.463 & 0.395 & 0.428 & 0.304 & 0.333 &   X   & 0.333 &  \\
2w0i & 0.333 & 0.463 & 0.428 & 0.463 & 0.463 & \cellcolor{fGreen!75}0.818 & \cellcolor{fGreen!25}0.666 & 0.463 & 0.5 & 0.621 & \cellcolor{fGreen!100}1.0 & 0.621 & \cellcolor{fGreen!50}0.714 & 0.333 &   X   &  \\


\end{tabular}}
\caption{Jaccard-Indizes der Aminos\"auregraphen. In den Zellen der Tabelle stehen die Jaccard-Indizes f\"ur die entsprechenden PDB-Dateien. F\"ur jede Zeile (jedes Protein) sind die 4 gr\"o"sten Indizes gr\"un unterlegt. Je dunkler das gr\"un ist, desto gr\"o"ser der Index}
\end{table}

\begin{table}
\label{table:occ-pg-tf}
\scalebox{0.8}{
\begin{tabular}{l l l l l l l l l l l l l l l l l}

PDB-ID & 1qpu & 1qq3 & 1cgn & 1he9 & 3gf9 & 1exs & 1ngl & 1qqs & 3slo & 1wjx & 5chy & 2id9 & 3i42 & 1d4o & 2w0i &  \\
1qpu &   X   & \cellcolor{fGreen!75}0.764 & \cellcolor{fGreen!100}0.764 & 0.578 & 0.463 & 0.463 & 0.621 & 0.621 & 0.428 & 0.578 & 0.666 & \cellcolor{fGreen!50}0.666 & \cellcolor{fGreen!25}0.666 & 0.538 & 0.395 &  \\
1qq3 & 0.764 &   X   & \cellcolor{fGreen!100}0.935 & 0.764 & 0.463 & 0.621 & 0.714 & 0.714 & 0.538 & 0.714 & \cellcolor{fGreen!25}0.764 & \cellcolor{fGreen!75}0.875 & \cellcolor{fGreen!50}0.818 & 0.666 & 0.5 &  \\
1cgn & \cellcolor{fGreen!25}0.764 & \cellcolor{fGreen!100}0.935 &   X   & 0.764 & 0.428 & 0.578 & 0.714 & 0.714 & 0.538 & 0.666 & 0.714 & \cellcolor{fGreen!75}0.875 & \cellcolor{fGreen!50}0.818 & 0.621 & 0.463 &  \\
1he9 & 0.578 & \cellcolor{fGreen!75}0.764 & \cellcolor{fGreen!100}0.764 &   X   & 0.463 & 0.578 & 0.621 & 0.621 & 0.578 & 0.538 & 0.621 & \cellcolor{fGreen!25}0.714 & \cellcolor{fGreen!50}0.714 & 0.666 & 0.538 &  \\
3gf9 & 0.463 & 0.463 & 0.428 & 0.463 &   X   & 0.428 & \cellcolor{fGreen!25}0.463 & 0.463 & 0.463 & \cellcolor{fGreen!50}0.463 & \cellcolor{fGreen!100}0.538 & 0.428 & 0.463 & \cellcolor{fGreen!75}0.5 & 0.333 &  \\
1exs & 0.463 & \cellcolor{fGreen!25}0.621 & 0.578 & 0.578 & 0.428 &   X   & 0.5 & 0.5 & \cellcolor{fGreen!100}0.714 & 0.5 & 0.578 & \cellcolor{fGreen!50}0.666 & 0.578 & \cellcolor{fGreen!75}0.714 & 0.621 &  \\
1ngl & 0.621 & 0.714 & 0.714 & 0.621 & 0.463 & 0.5 &   X   & \cellcolor{fGreen!100}1.0 & 0.538 & \cellcolor{fGreen!50}0.818 & \cellcolor{fGreen!25}0.818 & 0.818 & \cellcolor{fGreen!75}0.875 & 0.578 & 0.395 &  \\
1qqs & 0.621 & 0.714 & 0.714 & 0.621 & 0.463 & 0.5 & \cellcolor{fGreen!100}1.0 &   X   & 0.538 & \cellcolor{fGreen!25}0.818 & 0.818 & \cellcolor{fGreen!50}0.818 & \cellcolor{fGreen!75}0.875 & 0.578 & 0.395 &  \\
3slo & 0.428 & 0.538 & 0.538 & 0.578 & 0.463 & \cellcolor{fGreen!100}0.714 & 0.538 & 0.538 &   X   & 0.538 & 0.5 & \cellcolor{fGreen!25}0.578 & \cellcolor{fGreen!75}0.621 & \cellcolor{fGreen!50}0.578 & 0.463 &  \\
1wjx & 0.578 & 0.714 & 0.666 & 0.538 & 0.463 & 0.5 & \cellcolor{fGreen!50}0.818 & \cellcolor{fGreen!100}0.818 & 0.538 &   X   & \cellcolor{fGreen!25}0.764 & 0.714 & \cellcolor{fGreen!75}0.818 & 0.538 & 0.395 &  \\
5chy & 0.666 & 0.764 & 0.714 & 0.621 & 0.538 & 0.578 & \cellcolor{fGreen!75}0.818 & \cellcolor{fGreen!25}0.818 & 0.5 & 0.764 &   X   & \cellcolor{fGreen!100}0.875 & \cellcolor{fGreen!50}0.818 & 0.621 & 0.395 &  \\
2id9 & 0.666 & \cellcolor{fGreen!50}0.875 & \cellcolor{fGreen!25}0.875 & 0.714 & 0.428 & 0.666 & 0.818 & 0.818 & 0.578 & 0.714 & \cellcolor{fGreen!75}0.875 &   X   & \cellcolor{fGreen!100}0.935 & 0.714 & 0.463 &  \\
3i42 & 0.666 & 0.818 & 0.818 & 0.714 & 0.463 & 0.578 & \cellcolor{fGreen!50}0.875 & \cellcolor{fGreen!75}0.875 & 0.621 & \cellcolor{fGreen!25}0.818 & 0.818 & \cellcolor{fGreen!100}0.935 &   X   & 0.666 & 0.463 &  \\
1d4o & 0.538 & \cellcolor{fGreen!50}0.666 & 0.621 & 0.666 & 0.5 & \cellcolor{fGreen!100}0.714 & 0.578 & 0.578 & 0.578 & 0.538 & 0.621 & \cellcolor{fGreen!75}0.714 & \cellcolor{fGreen!25}0.666 &   X   & 0.5 &  \\
2w0i & 0.395 & \cellcolor{fGreen!50}0.5 & 0.463 & \cellcolor{fGreen!75}0.538 & 0.333 & \cellcolor{fGreen!100}0.621 & 0.395 & 0.395 & 0.463 & 0.395 & 0.395 & 0.463 & 0.463 & \cellcolor{fGreen!25}0.5 &   X   &  \\

\end{tabular}}
\caption{Jaccard-Indizes der Proteingraphen. In den Zellen der Tabelle stehen die Jaccard-Indizes f\"ur die entsprechenden PDB-Dateien. F\"ur jede Zeile (jedes Protein) sind die 4 gr\"o"sten Indizes gr\"un unterlegt. Je dunkler das gr\"un ist, desto gr\"o"ser der Index}
\end{table}


\begin{table}
\label{table:occ-cg-tc}
\scalebox{0.8}{
\begin{tabular}{l l l l l l l l l l l l l l l l l}

PDB-ID & 1qpu & 1qq3 & 1cgn & 1he9 & 3gf9 & 1exs & 1ngl & 1qqs & 3slo & 1wjx & 5chy & 2id9 & 3i42 & 1d4o & 2w0i &  \\
1qpu &   X   & \cellcolor{fGreen!100}0.428 & 0.395 & \cellcolor{fGreen!50}0.428 & 0.153 & 0.304 & 0.395 & 0.132 & 0.111 & 0.363 & 0.395 & \cellcolor{fGreen!75}0.428 & \cellcolor{fGreen!25}0.428 & 0.276 & 0.276 &  \\
1qq3 & 0.428 &   X   & \cellcolor{fGreen!100}0.666 & \cellcolor{fGreen!75}0.621 & 0.2 & 0.395 & 0.5 & 0.2 & 0.090 & 0.5 & 0.5 & \cellcolor{fGreen!50}0.538 & \cellcolor{fGreen!25}0.538 & 0.333 & 0.428 &  \\
1cgn & 0.395 & \cellcolor{fGreen!100}0.666 &   X   & \cellcolor{fGreen!75}0.621 & 0.224 & 0.428 & 0.5 & 0.224 & 0.111 & 0.463 & 0.5 & \cellcolor{fGreen!25}0.538 & \cellcolor{fGreen!50}0.538 & 0.363 & 0.363 &  \\
1he9 & 0.428 & 0.621 & 0.621 &   X   & 0.25 & 0.578 & \cellcolor{fGreen!50}0.621 & 0.153 & 0.090 & 0.538 & \cellcolor{fGreen!25}0.621 & \cellcolor{fGreen!100}0.714 & \cellcolor{fGreen!75}0.714 & 0.428 & 0.538 &  \\
3gf9 & 0.153 & 0.2 & 0.224 & 0.25 &   X   & \cellcolor{fGreen!75}0.276 & 0.224 & 0.153 & 0.153 & \cellcolor{fGreen!50}0.276 & \cellcolor{fGreen!100}0.304 & 0.25 & 0.224 & \cellcolor{fGreen!25}0.276 & 0.224 &  \\
1exs & 0.304 & 0.395 & 0.428 & 0.578 & 0.276 &   X   & 0.5 & 0.2 & 0.153 & \cellcolor{fGreen!25}0.578 & 0.578 & \cellcolor{fGreen!100}0.666 & \cellcolor{fGreen!50}0.578 & 0.5 & \cellcolor{fGreen!75}0.621 &  \\
1ngl & 0.395 & 0.5 & 0.5 & 0.621 & 0.224 & 0.5 &   X   & 0.132 & 0.132 & \cellcolor{fGreen!25}0.818 & \cellcolor{fGreen!75}0.818 & \cellcolor{fGreen!50}0.818 & \cellcolor{fGreen!100}0.875 & 0.428 & 0.395 &  \\
1qqs & 0.132 & \cellcolor{fGreen!50}0.2 & \cellcolor{fGreen!100}0.224 & 0.153 & 0.153 & \cellcolor{fGreen!25}0.2 & 0.132 &   X   & \cellcolor{fGreen!75}0.224 & 0.132 & 0.132 & 0.132 & 0.132 & 0.132 & 0.2 &  \\
3slo & 0.111 & 0.090 & 0.111 & 0.090 & \cellcolor{fGreen!25}0.153 & \cellcolor{fGreen!50}0.153 & 0.132 & \cellcolor{fGreen!100}0.224 &   X   & 0.111 & 0.090 & 0.090 & 0.111 & 0.111 & \cellcolor{fGreen!75}0.153 &  \\
1wjx & 0.363 & 0.5 & 0.463 & 0.538 & 0.276 & 0.578 & \cellcolor{fGreen!50}0.818 & 0.132 & 0.111 &   X   & \cellcolor{fGreen!100}0.935 & \cellcolor{fGreen!25}0.714 & \cellcolor{fGreen!75}0.818 & 0.5 & 0.395 &  \\
5chy & 0.395 & 0.5 & 0.5 & 0.621 & 0.304 & 0.578 & \cellcolor{fGreen!25}0.818 & 0.132 & 0.090 & \cellcolor{fGreen!100}0.935 &   X   & \cellcolor{fGreen!75}0.875 & \cellcolor{fGreen!50}0.818 & 0.5 & 0.395 &  \\
2id9 & 0.428 & 0.538 & 0.538 & 0.714 & 0.25 & 0.666 & \cellcolor{fGreen!50}0.818 & 0.132 & 0.090 & \cellcolor{fGreen!25}0.714 & \cellcolor{fGreen!75}0.875 &   X   & \cellcolor{fGreen!100}0.935 & 0.5 & 0.463 &  \\
3i42 & 0.428 & 0.538 & 0.538 & 0.714 & 0.224 & 0.578 & \cellcolor{fGreen!75}0.875 & 0.132 & 0.111 & \cellcolor{fGreen!25}0.818 & \cellcolor{fGreen!50}0.818 & \cellcolor{fGreen!100}0.935 &   X   & 0.428 & 0.463 &  \\
1d4o & 0.276 & 0.333 & 0.363 & 0.428 & 0.276 & \cellcolor{fGreen!100}0.5 & 0.428 & 0.132 & 0.111 & \cellcolor{fGreen!50}0.5 & \cellcolor{fGreen!25}0.5 & \cellcolor{fGreen!75}0.5 & 0.428 &   X   & 0.333 &  \\
2w0i & 0.276 & 0.428 & 0.363 & \cellcolor{fGreen!75}0.538 & 0.224 & \cellcolor{fGreen!100}0.621 & 0.395 & 0.2 & 0.153 & 0.395 & 0.395 & \cellcolor{fGreen!25}0.463 & \cellcolor{fGreen!50}0.463 & 0.333 &   X   &  \\

\end{tabular}}
\caption{Jaccard-Indizes der Komplexgraphen. In den Zellen der Tabelle stehen die Jaccard-Indizes f\"ur die entsprechenden PDB-Dateien. F\"ur jede Zeile (jedes Protein) sind die 4 gr\"o"sten Indizes gr\"un unterlegt. Je dunkler das gr\"un ist, desto gr\"o"ser der Index}
\end{table}


\begin{table}
\label{tab:correlations}
\begin{tabular}{c c c c}
Datensatz & AAG     &     PG &     CG \\
AAG       &  1      & 0.6911 & 0.7656 \\
PG        &  0.6911 &      1 & 0.6939 \\
CG        &  0.6939 & 0.6939 & 1      \\

\end{tabular}
\caption{Korrelationen der \"Ahnlichkeitsbewertungen der verschiedenen Graphformate}
\end{table}




\bibliographystyle{plain}
\bibliography{literatur}



\end{document}