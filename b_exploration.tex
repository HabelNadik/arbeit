\documentclass{article}
\author{Ben Haladik}
\title{Bachelorarbeit: Explorationsphase}
\usepackage{mathtools}
\usepackage{ngerman}
\usepackage{hyperref}

\begin{document}

\pagenumbering{gobble}
\maketitle
\newpage

\pagenumbering{arabic}
\tableofcontents
\newpage

\section{Wichtige Fragen}

Um die Bachelorarbeit gut schreiben zu k\"onnen sind einige wichtige Fragen zu beantworten.
Zur Erinnerung der Titel der Arbeit soll lauten: \\
\textbf{Bioinformatische Anwendung von Graphlets auf Proteinstrukturtopologiegraphen zur \"Ahnlichkeitsanalyse von Proteinen}. \\
Wie bei vielen naturwissenschaftlichen Arbeiten ist er sehr sperrig und passt nicht in eine Zeile.
Folgende Fragen sind f\"ur die Implementierung zu beantworten:

\paragraph{Welche Metrik} soll ich benutzen, um die berechneten Vektoren zu vergleichen? Hier bietet es sich an in die Liste vorhandener Strukturalignment-Methoden zu schauen, was direkt die n\"achsten zwei Fragen liefert:
\paragraph{Wie clustere} ich die Vektoren?

\paragraph{Welche Methoden} soll ich benutzen, um sie mit dem \texttt{graphletAnalyser} zu vergleichen?
An sich w\"are es klug, diese drei Fragen zusammen zu beantworten. Bei der Recherche nach Strukturalignment-Methoden werde ich bestimmt auf einige Datens\"atze sto"sen, die hofffentlich \"offentlich verf\"ugbar sind. Ich sollte auch eine Idee davon bekommen, welche Metriken in der Bioinformatik \"ublicherweise benutzt werden. Die gro"se Frage ist nat\"urlich und leider die, die von dieser Arbeit nicht beantwort werden kann:

\textbf{Was ist ein biologisch sinnvolles Strukturalignment?} \\

Dieser Frage will ich mich im Laufe der Arbeit n\"ahern. Ich bin so aufgeregt!



\section{Datensatz}

\subsection{Proteine von Tim}

\paragraph{4-Helix-Bundles}
Cytochrome b562 (PDB: 1QPU) und human growth hormone (PDB: 1HGU)

\paragraph{Globin-Fold}
Hier gibt es viele Proteine mit niedriger Sequenzidentit\"at und hoher struktureller \"Ahnlichkeit. Gute Kandidaten sind Hemoglobin, Myoglobine und Phycocyanine

\paragraph{TIM-Barrels}
Die Pyruvat-Kinase hat eine TIM-Barrel-Dom\"ane (PDB: 1A3W). Des weiteren wurde die Triosephosphat-Isomerase vorgeschlagen (7TIM).

\subsection{Proteine aus der Literatur}


Ein Artikel mit dem Titel: \textit{Protein Structure Comaprison by Alignment of Distance Matrices} (URL: \url{http://www.sciencedirect.com/science/article/pii/S0022283683714890}) scheint laut Abstract gute Kandidaten liefern zu k\"onnen. \\
Ein weiterer Artikel behandelt die Frage, ob es eine einziges korrektes Strukturalignment gibt/geben kann und ist unter der URL: \url{http://onlinelibrary.wiley.com/doi/10.1002/pro.5560050711/abstract} zu finden. \\
Der Artikel: \textit{Structural alignment of proteins by a novel TOPOFIT method, as a superimposition of common volumes at a topomax point} unter der URL: \url{http://onlinelibrary.wiley.com/doi/10.1110/ps.04672604/full}. beschreibt eine Strukturalignment-Methode, die  versucht die Substrukturen mit der minimalen Abweichung zu finden.

\subsection{Datenbanken und gro"se Datens\"atze f\"ur die Analyse}

Bisher scheinen sich zwei gro"se Datenbanken f\"ur dei Analyse zu eignen.

\paragraph{ASTRAL} ist ein Datensatz, der prinzipiell zu SCOPe geh\"ort. In ihm werden - basierend auf Sequenzen - Proteine mit gro"ser \"Ahnlichkeit aufbewahrt, die nur geringe bzw. keine Homologie aufweisen und gro"se strukturelle \"Ahnlichkeit besitzen.

\paragraph{SISYPHUS} ist ein Datensatz der ebenfalls mit SCOPe assoziiert ist.
Hierin befinden sich Proteine mit \textit{nicht-trivialen Beziehungen} , die zusammen gruppiert werden.
Hierzu geh\"oren beispielsweise Proteine, die sich durch zyklische Vertauschungen unterschieden, oder sogenannte Cham\"aleon-Sekund\"arstrukturen aufweisen, die sich je nach Umgebung \"andern.
Der Datensatz ist unter der URL:
\url{http://www.spice-3d.org/sisyphus/index.jsp} zu erreichen.
Leider wird er seit etwa 2009 nicht mehr aktualisiert. 

\subsection{Proteinwahl anhand von EC-Termen}
Es k\"onnte eine gute Idee sein, anhand von EC-Termen Proteine auszuw\"ahlen. Die Methode kann getestet werden, indem man zwei oder mehrere verschiedene Gruppen von Proteinen w\"ahlt, die jeweils mit dem selben EC-Term assoziiert sind, dabei aber geringe Homolgien aufweisen.
Wenn die Proteine, die mit den selben EC-Termen assoziiert sind, im selben Cluster landen, ist die Methode sinnvoll. Dies folgt dem Dogma, das die Funktion eines Proteins aus der Struktur folgt.

CATH erm\"oglicht die Suche nach EC-Termen. Nach den ersten Tests k\"onnte es sinnvoll sein, automatisiert zu suchen. Bisher stellt CATH seine API jedoch noch nicht zur Verf\"ugung. Vielleicht ist ein Umweg \"uber die PDB m\"oglich.

\subsection{Proteine, die in Papern verwendet wurden}

In dem Artikel: \textit{Protein strcutre alignment by incremental combinatorial extension (CE) of the optimal path} wird ebenfalls ein Algorithmus f\"ur Strukturalignment vorgestellt. 

\section{Methoden}
Der Artikel \textit{Advances and pitfalls of protein structure alignment} (URL: \url{http://www.sciencedirect.com/science/article/pii/S0959440X09000621}) scheint ein guter Punkt f\"ur den Anfang zu sein.
In ihm werden verschiedene Strukturalignment-Methoden verglichen und bewertet! GEILOMAT!!!!


Den Algorithmus DALI k\"onnte man sich genauer anschauen. Vielleicht taugt er zum Vergleich mit dem graphletAnalyser

\subsection{Datengewinnung aus der PTGL}

Die Daten aus der PTGL werden mit bash Skripten unter Verwendung der REST API (URL: \url{http://ptgl.uni-frankfurt.de/api/}) gewonnen

\subsection{Pipeline}

Sch\"on w\"are es, eine Pipeline zu haben, die die Suche nach \textit{Graphlets} automatisiert. Der Workflow k\"onnte folgenderma"sen aussehen.
\begin{enumerate}

\item Die Pipeline erh\"alt EC-Terme und eine Zahl x, die angibt, wie viele Proteine für jeden EC-Term gew\"ahlt werden sollen. Dann durchsucht sie eine Datenbank. Dies wird wahrscheinlich GO werden, da CATH noch keine API zur Verf\"ugung stellt. Von dort werden die ersten x PDB-IDs extrahiert.

\item Mit Hilfe der REST API werden aus der PTGL die Proteingraphen extrahiert, deren PDB-IDs zuvor gesammelt wurden.

\item Die Proteingraphen werden dem \texttt{graphletAnalyser} \"ubergeben. Dieser berechnet die \textit{Graphlets} und clustert sie anhand einer Metrik, die noch bestimmt werden muss.

\section{Literaturliste}

Hier stehen die URLs von Papern, die ben\"otigt wurden.

Advances and Pitfalls of Protein Structure Alignment: \\
\url{http://www.sciencedirect.com/science/article/pii/S0959440X09000621} \\

Paper zur PTGL: \\
\url{{http://nar.oxfordjournals.org/content/38/suppl_1/D326.abstract}} \\
\url{http://drops.dagstuhl.de/opus/volltexte/2012/3722}

\end{enumerate}


\end{document}